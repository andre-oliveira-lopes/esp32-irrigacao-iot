% --- textbf = mantem o texto em negrito. 
\section*{\fontsize{12pt}{14pt}\selectfont\textbf{INTRODUÇÃO}} % 
\addcontentsline{toc}{section}{INTRODUÇÃO}  % colocar no sumário
\vspace*{-0.7\baselineskip}  % Remove o espaço extra entre título e texto

A agricultura familiar desempenha um papel fundamental na produção de alimentos no Brasil, mas ainda enfrenta grandes desafios relacionados ao uso eficiente da água, à dependência de mão de obra e à ausência de tecnologias acessíveis. Em pequenas propriedades e pomares, a irrigação é frequentemente realizada de forma manual e imprecisa, o que resulta em desperdício de recursos, baixa produtividade e esforço físico elevado por parte dos agricultores. Nesse cenário, a integração de tecnologias de Internet das Coisas (IoT) apresenta-se como uma alternativa viável e promissora para automatizar processos, otimizar o uso da água e melhorar a gestão hídrica nas lavouras.

Este trabalho propõe o desenvolvimento de um sistema automatizado de irrigação, voltado especialmente para pequenos e médios cultivos, utilizando o microcontrolador ESP32 DevKit V1 em conjunto com sensores ambientais, multiplexadores, relés e válvulas solenoides. O sistema permite o monitoramento em tempo real da umidade do solo, luminosidade, temperatura e umidade do ar, com tomada de decisão automática baseada em dados sensoriais. A comunicação com a plataforma Arduino IoT Cloud possibilita o controle remoto das funções por meio de dashboards online, bem como a integração com assistentes virtuais como a Alexa.

Além de funcionalidades como o modo de economia de energia (deep sleep) e a previsão de alimentação por energia solar, o sistema também conta com mecanismos de envio de dados para o Google Sheets e integração com o Power BI para análise e visualização. A solução se destaca por ser acessível, escalável e de fácil adaptação, sendo projetada para operar em regiões com infraestrutura limitada. Ao aliar automação, baixo custo e sustentabilidade, o projeto visa promover uma agricultura mais eficiente, tecnológica e voltada à realidade dos pequenos produtores.