% CAPÍTULO 1 – PROBLEMA

\vspace*{-1.0em} % Ajuste conforme visual pra iniciar "1 PROBLEMA". 
\section*{\fontsize{12pt}{14pt}\selectfont\textbf{1. PROBLEMA}}
\addcontentsline{toc}{section}{1. PROBLEMA}
\vspace{-0.7\baselineskip}

\section*{\fontsize{12pt}{14pt}\selectfont\textbf{1.1. Tema do Trabalho}}
\addcontentsline{toc}{section}{1.1. Tema do Trabalho}
\vspace{-0.7\baselineskip}

A situação-problema abordada refere-se à irrigação, uma técnica essencial para o suprimento hídrico das plantas, que, quando realizada por métodos tradicionais, pode resultar em uso ineficiente da água, desperdícios e escassez desse recurso vital, por isso o presente tema escolhido foi; Irrigação Automatizada em Pomares e Pequenas Plantações: Uma Abordagem Baseada em Integração IoT na Plataforma Arduino Cloud.

\vspace*{-1.0em} % Ajuste 
\section*{\fontsize{12pt}{14pt}\selectfont\textbf{1.2. Contextualização}}
\addcontentsline{toc}{section}{1.2. Contextualização}
\vspace{-0.7\baselineskip}

A agricultura familiar representa uma parcela significativa da produção agrícola brasileira, especialmente em regiões onde a mecanização ainda é limitada. Em pequenas propriedades e pomares, o manejo hídrico costuma ser realizado de forma manual, exigindo longas jornadas de trabalho e gerando, muitas vezes, desperdício de água.

Com o avanço das tecnologias de Internet das Coisas (IoT), torna-se viável implementar soluções automatizadas e acessíveis que otimizem recursos, aumentem a produtividade e melhorem a qualidade de vida dos agricultores. O sistema proposto destina-se, portanto, a pequenos produtores rurais, com foco especial em propriedades que não possuem acesso a tecnologias caras ou infraestrutura avançada.

\vspace*{-1.0em} % Ajuste 
\section*{\fontsize{12pt}{14pt}\selectfont\textbf{1.3. Situação-Problema}}
\addcontentsline{toc}{section}{1.3. Situação-Problema}
\vspace{-0.7\baselineskip}

A irrigação manual e imprecisa nas pequenas propriedades leva ao uso ineficiente da água, impactando diretamente a produtividade e a sustentabilidade da lavoura. Além disso, o esforço físico constante exigido para essa tarefa sobrecarrega os trabalhadores, especialmente em famílias com pouca mão de obra disponível. A ausência de controle em tempo real e de automação impede decisões rápidas e assertivas, comprometendo o rendimento das colheitas e o bem-estar dos produtores.

\vspace*{-1.0em} % Ajuste 
\section*{\fontsize{12pt}{14pt}\selectfont\textbf{1.4. Breve Descrição da Solução}}
\addcontentsline{toc}{section}{1.4. Breve Descrição da Solução}
\vspace{-0.7\baselineskip}

Foi desenvolvido um sistema automatizado de irrigação baseado em tecnologia IoT, utilizando sensores ambientais integrados à plataforma Arduino Cloud devido à compatibilidade com o protocolo MQTT e à placa ESP32 DevKit V1. A solução permite o desenvolvimento de um dashboard para visualização em tempo real de umidade do solo, luminosidade, temperatura e umidade do ar e outros, acionando automaticamente válvulas solenoides conforme a necessidade.

Os dados são armazenados no Google Sheets e podem ser analisados posteriormente via Power BI. A integração com a assistente virtual Alexa amplia a acessibilidade, facilitando o uso do sistema mesmo por pessoas com pouco domínio tecnológico. O projeto visa aumentar a autonomia, reduzir o desperdício de água e melhorar a qualidade de vida do agricultor familiar.