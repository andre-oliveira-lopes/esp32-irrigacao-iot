% CAPÍTULO 2 – DESENVOLVIMENTO

\vspace*{-1.0em}  % Ajuste 
\section*{\fontsize{12pt}{14pt}\selectfont\textbf{2. CONTEXTUALIZAÇÃO DO PROBLEMA}}
\addcontentsline{toc}{section}{2 CONTEXTUALIZAÇÃO DO PROBLEMA}
\vspace{-0.7\baselineskip}

A irrigação é uma técnica essencial para o cultivo de diversas culturas, especialmente em regiões sujeitas a estiagens prolongadas. Entretanto, métodos tradicionais de irrigação, como o uso manual de mangueiras ou baldes, ainda são amplamente utilizados por agricultores familiares, o que resulta em desperdício de água, baixa eficiência e alta dependência de mão de obra. Tais práticas comprometem a produtividade e a sustentabilidade das lavouras.

Com a crescente escassez de recursos hídricos e os desafios enfrentados no campo, a automação do processo de irrigação, por meio da integração com tecnologias de Internet das Coisas (IoT), surge como uma solução promissora. Essa abordagem permite não apenas a economia de água, mas também a otimização do tempo de trabalho e a redução do esforço físico exigido dos agricultores.

\needspace{5\baselineskip} % reserva espaço suficiente para a próxima seção
\vspace*{-1.0em}  % Ajuste
\section*{\fontsize{12pt}{14pt}\selectfont\textbf{2.1. Premissas e Restrições do Projeto}}
\addcontentsline{toc}{section}{2.1 Premissas e Restrições do Projeto}
\vspace{-0.7\baselineskip}

Para a viabilização do sistema proposto, algumas premissas foram estabelecidas:

\listarecuada{
    \item Acesso estável à internet na área de operação.
    \item Disponibilidade de energia elétrica, ou fontes alternativas como placas solares.
    \item Espaço físico adequado para testes e instalação de sensores ambientais.
    \item Precisão satisfatória de sensores de baixo custo.
}

Contudo, o projeto também enfrenta algumas restrições importantes:

\listarecuada{
  \setcounter{enumi}{4}  % Começa do item 5
  \item Orçamento limitado, o que restringe o uso de equipamentos sofisticados.
  \item Manutenção frequente dos sensores, exigindo suporte contínuo.
  \item Calibração dos sensores que pode ser complexa e dependente de ajustes manuais no código.
  \item Instabilidades de conectividade em áreas rurais com sinal fraco.
  \item Integração dos módulos IoT, que demandou diversas adaptações técnicas.
  \item Gestão do consumo energético, especialmente em cenários com adoção de fontes renováveis.
}

Portanto, esses são as premissas e algumas restrições do projeto.

\needspace{5\baselineskip} % reserva espaço suficiente para a próxima seção
\vspace*{-1.0em}  % Ajuste
\section*{\fontsize{12pt}{14pt}\selectfont\textbf{2.2. Caracterização da Empresa}}
\addcontentsline{toc}{section}{2.2 Caracterização da Empresa}

\needspace{5\baselineskip} % reserva espaço suficiente para a próxima seção
\vspace*{-1.0em}  % Ajuste
\section*{\fontsize{12pt}{14pt}\selectfont\textbf{2.2.1. Histórico da Empresa}}
\addcontentsline{toc}{section}{2.2.1. Histórico da Empresa}
\vspace{-0.7\baselineskip}

A AgroIoT Brasil é uma startup fictícia criada em 2025, nascida do ambiente acadêmico, com o propósito de democratizar o acesso à automação agrícola entre pequenos e médios produtores. A iniciativa surgiu da observação das dificuldades enfrentadas por agricultores familiares, principalmente no que diz respeito ao manejo manual da irrigação e à falta de tecnologias acessíveis no campo. 

A empresa iniciou suas atividades com uma equipe reduzida, composta por profissionais das áreas de engenharia, agronomia e tecnologia da informação, atuando em regime de colaboração.

\needspace{5\baselineskip} % reserva espaço suficiente para a próxima seção
\vspace*{-1.0em}  % Ajuste
\section*{\fontsize{12pt}{14pt}\selectfont\textbf{2.2.2. Atividades da Empresa}}
\addcontentsline{toc}{section}{2.2.2. Atividades da Empresa}
\vspace{-0.7\baselineskip}

A AgroIoT Brasil dedica-se à pesquisa, desenvolvimento e comercialização de kits de irrigação automatizada com monitoramento remoto, compostos por:

\listarecuada{
    \item Sensores de umidade do solo, luminosidade, temperatura e umidade do ar e outros.
    \item Módulos de conectividade (ESP32 + Arduino Cloud via MQTT).
    \item Dashboards para visualização de dados antigos e em tempo real, além de transmissão continua via bluetooth.
    \item Integração com assistente virtual Alexa.
    \item Setor Administrativo e Financeiro.
}

Além da oferta dos kits, a empresa também fornece serviços como instalação, suporte técnico e treinamento prático para os usuários.

\needspace{5\baselineskip} % reserva espaço suficiente para a próxima seção
\vspace*{-1.0em}  % Ajuste
\section*{\fontsize{12pt}{14pt}\selectfont\textbf{2.2.3. Mercado Consumidor}}
\addcontentsline{toc}{section}{2.2.3. Mercado Consumidor}
\vspace{-0.7\baselineskip}

O público-alvo da AgroIoT Brasil inclui:

\listarecuada{
    \item Agricultores familiares.
    \item Horticultores urbanos.
    \item Cooperativas agroecológicas.
    \item Proprietários de pequenas e médias propriedades rurais.
    \item Instituições de ensino técnico com foco em práticas agrícolas.
}
Esses consumidores compartilham características como limitação de recursos financeiros, escassez de mão de obra e interesse crescente por soluções tecnológicas de fácil adoção.

\needspace{5\baselineskip} % reserva espaço suficiente para a próxima seção
\vspace*{-1.0em}  % Ajuste
\section*{\fontsize{12pt}{14pt}\selectfont\textbf{2.2.4. Concorrência}}
\addcontentsline{toc}{section}{2.2.4. Concorrência}
\vspace{-0.7\baselineskip}

No mercado brasileiro, empresas como Netafim e Toro dominam o setor de irrigação de precisão com soluções voltadas para grandes lavouras e projetos agrícolas de grande escala. Em contraste, a AgroIoT Brasil foca em atender um nicho ainda pouco explorado: o pequeno e médio produtor rural. Sua proposta de valor reside em: 

\listarecuada{
    \item Soluções personalizadas e de baixo custo.
    \item Suporte técnico acessível.
    \item Capacitação com linguagem didática e aplicação prática.
}

\needspace{5\baselineskip} % reserva espaço suficiente para a próxima seção
\vspace*{-1.0em}  % Ajuste
\section*{\fontsize{12pt}{14pt}\selectfont\textbf{2.2.5. Organograma}}
\addcontentsline{toc}{section}{2.2.5. Organograma} 
\vspace{-0.7\baselineskip}

A estrutura organizacional da AgroIoT Brasil é enxuta e baseada em metodologias ágeis. A empresa se divide em cinco núcleos principais:

\listarecuada{
    \item Diretoria Executiva e Estratégica – responsável pela tomada de decisões e definição de metas.
    \item Departamento de Engenharia e Desenvolvimento – voltado à criação e teste de novas soluções.
    \item Setor Comercial e Marketing – encarregado da divulgação e relacionamento com clientes.
    \item Atendimento ao Cliente e Suporte Técnico – apoio contínuo aos usuários.
    \item Setor Administrativo e Financeiro – gestão de recursos e planejamento financeiro.
}

A estrutura colaborativa permite flexibilidade, inovação contínua e forte aproximação com o público atendido.

\needspace{5\baselineskip} % reserva espaço suficiente para a próxima seção
\vspace*{-1.0em}  % Ajuste
\section*{\fontsize{12pt}{14pt}\selectfont\textbf{2.3. Proposta de Trabalho}}
\addcontentsline{toc}{section}{2.3. Proposta de Trabalho} 

\needspace{5\baselineskip} % reserva espaço suficiente para a próxima seção
\vspace*{-1.0em}  % Ajuste
\section*{\fontsize{12pt}{14pt}\selectfont\textbf{2.3.1. Método do Trabalho}}
\addcontentsline{toc}{section}{2.3.1. Método do Trabalho} 
\vspace{-0.7\baselineskip}

O desenvolvimento do projeto seguirá uma abordagem de pesquisa aplicada com ênfase em metodologia experimental, utilizando o modelo incremental com práticas de desenvolvimento ágil (Scrum). Isso permitirá validações contínuas por meio de ciclos curtos de desenvolvimento, testes práticos e refinamento da solução com base no feedback obtido em campo. 

\selectfont\textbf{Levantamento de dados:}

Os dados necessários para o desenvolvimento do sistema foram (ou serão) obtidos por meio de:

\listarecuada{
    \item Entrevistas com agricultores familiares, para entender as principais dificuldades enfrentadas com a irrigação manual.
    \item Observação direta em pequenas propriedades rurais, analisando padrões de irrigação, tipo de cultivo e uso atual de tecnologias.
    \item Pesquisa bibliográfica em fontes acadêmicas, artigos técnicos e manuais sobre sensores de umidade, automação agrícola e sistemas IoT.
}

\selectfont\textbf{Metodologia de desenvolvimento:}

Será utilizado o modelo incremental com base em Scrum, permitindo:

\listarecuada{
    \item Divisão do projeto em sprints quinzenais.
    \item Autoavaliação de execução individual.
    \item Protótipos evolutivos com incrementos funcionais a cada ciclo.
}

\selectfont\textbf{Ferramentas e técnicas de modelagem:}

\listarecuada{
    \item Diagramas de caso de uso, sequência e classes (UML) para modelagem dos requisitos e funcionalidades.
    \item Arduino IDE para desenvolvimento do código embarcado.
    \item Arduino Cloud para interface com a nuvem e dashboards de visualização.
    \item Fritzing para o protótipo esquemático do circuito eletrônico (Anexo II).
    \item TinkerCAD com YouTube para simulações iniciais do projeto.
    \item Comandos via Alexa para a acessibilidade das funções.
}

\needspace{5\baselineskip} % reserva espaço suficiente para a próxima seção
\vspace*{-1.0em}  % Ajuste
\section*{\fontsize{12pt}{14pt}\selectfont\textbf{2.3.2. Previsão e Alocação de Recursos (Humanos e Materiais)}}
\addcontentsline{toc}{section}{2.3.2. Previsão e Alocação de Recursos (Humanos e Materiais)} 
\vspace{-0.7\baselineskip}

\selectfont\textbf{Recursos humanos:}

\listarecuada{
    \item Desenvolvedor Full Stack / Maker (responsável por programação, testes, integração com nuvem).
    \item Engenheiro eletrônico ou técnico em eletrônica (auxílio com sensores, circuitos e calibração).
    \item Especialista de campo (auxílio na aplicação em propriedade rural e coleta de feedback).
    \item Coordenador do projeto (o próprio autor, responsável pelo planejamento, cronograma e documentação).
    \item Apoio eventual de professores orientadores e colegas da área agrícola.
}

\selectfont\textbf{Recursos materiais:}

\listarecuada{
    \item Placas ESP32 DevKit V1.  
    \item Esp32 Base Board.
    \item Sensores de umidade do solo.
    \item Sensores de Temperatura e umidade do ar (DHT11).
    \item Luminosidade (LDR).
    \item Módulos relé de 5v.
    \item Válvulas solenoides 12V.
    \item Multiplexador CD74HC4067.
    \item Painéis solares 50W / 12V + Reguladores de carga de 20A a 30A. 
    \item Bateria selada ou Li-Ion de 8A. 
    \item Fontes de alimentação de 12v e 5v para os sensores.
    \item Display LCD I2C 16x2. 
    \item Conectores de Derivação Emenda Wago: Modelo 221 ou 222.
    \item Fios bipolar paralelo cristal de 1,5mm² ou 2mm²
    \item Caso compre um relé com transistor ligado de forma diferente, você vai precisar de: 
    \begin{enumerate}
        \item Outro transistor, ele é o BC547.
        \item Um resistor de 10k Ohms.
        \item Placa De Circuito Perfurada Fenolite (O tamanho depende da necessidade).  
    \end{enumerate}
    \item Caixas de proteção IP65 para uso externo.
    \item Canos de 20 ou 25 mm com conectores (dependendo da necessidade).
    \item Ferramentas de desenvolvimento (gratuitas ou open-source).
    \item Conta ativa na plataforma Arduino Cloud.
    \item Ferramentas como: Alicate de ponta e de corte, chave de fenda e estrela, multímetro, chave de teste elétrico, filtro de linhas com várias tomadas para testes, estilete, cabo extensor de USB de qualidade, são opcionais, mas ajudaram no decorrer da construção.
}

\vspace*{-1.0em}  % Ajuste
\section*{\fontsize{12pt}{14pt}\selectfont\textbf{2.3.3. Cronograma de Trabalho (Diagrama de Gantt)}}
\addcontentsline{toc}{section}{2.3.3. Cronograma de Trabalho (Diagrama de Gantt)} 
\vspace{-0.7\baselineskip}

O projeto foi planejado para ocorrer em 6 meses, conforme abaixo:

\minhaimagemcomfonte{12cm}{capitulos/img/2_3_3_Cronograma_gantt}{Cronograma de Execução do Projeto}

\needspace{5\baselineskip} % reserva espaço suficiente para a próxima seção
\vspace*{-1.0em}  % Ajuste
\section*{\fontsize{12pt}{14pt}\selectfont\textbf{2.3.4. Previsão Orçamentária}}
\addcontentsline{toc}{section}{2.3.4. Previsão Orçamentária} 
\vspace{-0.7\baselineskip}

Com base nos recursos materiais de automação mencionados, estima-se um investimento mínimo de R\$ 1.160,00, distribuído da seguinte forma:

\begin{table}[H]
\centering
\captionsetup{justification=centering}
\caption{\textbf{Distribuição estimada de custos dos materiais do projeto}}
\vspace{-0.5em}
\footnotesize
\renewcommand{\arraystretch}{1.2} % espaçamento entre linhas

\begin{tabularx}{\textwidth}{>{\raggedright\arraybackslash}X c c}
\toprule
\textbf{ITEM} & \textbf{UNI.} & \textbf{PREÇO MÉDIO (BRL)} \\
\midrule
Placa ESP32 com Wi-Fi, Bluetooth ESP32S IDE Dual Core - Dev Kit V1 + Cabo Micro USB & 1 & R\$ 90,00 \\
Placa de Expansão para ESP32-DevKit V1 30 Pinos (Base Board) & 1 & R\$ 133,00 \\
Sensores de umidade do solo & 1 ou mais & R\$ 80,00 \\
Sensores de temperatura e umidade do ar (DHT11) & 1 & R\$ 22,00 \\
Módulo Sensor de Luminosidade Fotoresistor LDR & 1 & R\$ 10,00 \\
Módulos relé 1 canal de 5V & 1 ou mais & R\$ 14,00 \\
Válvulas solenoides 12V & 1 ou mais & R\$ 55,00 \\
Multiplexador CD74HC4067 & 1 & R\$ 16,00 \\
Painéis solares 50W / 12V + Regulador de carga de 20A & 1 & R\$ 120,00 \\
Painéis solares 50W / 12V (adicional) & 1 & R\$ 70,00 \\
Bateria selada 12V / 20Ah & 1 & R\$ 300,00 \\
Fonte de alimentação de 12V 2A para sensores & 1 & R\$ 40,00 \\
Fonte de alimentação de 5V 2A para sensores & 1 & R\$ 30,00 \\
Display LCD I2C 16x2 & 1 & R\$ 40,00 \\
Caixas de proteção IP65 para uso externo & 1 & R\$ 100,00 \\
Conta ativa na plataforma Arduino Cloud & 1 & R\$ 40,00 \\
Ferramentas de desenvolvimento (open-source) & 1 & Gratuito \\
\bottomrule
\end{tabularx}

\vspace{0.5em}
{\footnotesize\textit{Fonte: Elaborado pelo autor.}\par}
\end{table}

Com base nos recursos materiais elétricos e de estrutura para a construção do projeto em ambiente rural (Não incluso no valor acima por ser algo escalável):

\begin{table}[H]
\centering
\captionsetup{justification=centering}
\caption{\textbf{Materiais adicionais para montagem e instalação do sistema}}
\vspace{-0.5em}
\footnotesize
\renewcommand{\arraystretch}{1.2}

\begin{tabularx}{\textwidth}{>{\raggedright\arraybackslash}X c c}
\toprule
\textbf{ITEM} & \textbf{UNI.} & \textbf{PREÇO MÉDIO (BRL)} \\
\midrule
Canos de 20 ou 25 mm (PVC) & 1 ou mais & R\$ 20,00 à 40,00 \\
Conectores de Derivação Emenda Wago: 221 ou 222 & 1 ou mais & R\$ 8,00 \\
Fios bipolar paralelo cristal de 1{,}5mm² ou 2mm² & 1m & R\$ 7,00 \\
Conectores para canos & 1 ou mais & R\$ 5,00 à 20,00 \\
\bottomrule
\end{tabularx}

\vspace{0.5em}
{\footnotesize\textit{Fonte: Elaborado pelo autor.}\par}
\end{table}

Observa-se que na coluna “UNI” tem algo de diferente, como o dado de “1 ou mais”, que indica que a dimensão do projeto pode ser escalonada pelo produtor cliente. Abaixo é mostrado na tabela, em um grau energético, como isso seria dimensionado em capacidade de baterias e quantidade de placas solares:

\begin{table}[H]
\centering
\captionsetup{justification=centering}
\caption{\textbf{Dimensionamento energético estimado conforme a quantidade de plantas automatizadas}}
\vspace{-0.5em}
\scriptsize  % Fonte menor que footnotesize (~9pt)
\renewcommand{\arraystretch}{1.2}

\begin{tabularx}{\textwidth}{>{\centering\arraybackslash}m{1.0cm}
                                >{\centering\arraybackslash}m{1.2cm}
                                >{\centering\arraybackslash}m{1.2cm}
                                >{\centering\arraybackslash}m{2.0cm}
                                >{\centering\arraybackslash}m{1.6cm}
                                >{\centering\arraybackslash}m{1.6cm}
                                >{\centering\arraybackslash}m{1.0cm}
                                >{\centering\arraybackslash}m{1.6cm}
                                >{\centering\arraybackslash}m{1.2cm}}
\toprule
\textbf{Plantas (Kit)} & \textbf{Consumo (Ah/dia)} & \textbf{Bateria Mín. (Ah)} & \textbf{Capacidade Painel solar} & \textbf{Sobra c/ 1x 50W} & \textbf{Sobra c/ 2x 50W} & \textbf{Descarga} & \textbf{Controlador} & \textbf{Horas} \\
\midrule
1   & 5{,}21  & 10{,}42 & 1 painel de 50W   & +15{,}59 Ah & +35{,}99 Ah & 50\% & PWM 20A & 14h \\
2   & 5{,}41  & 10{,}82 & 1 painel de 50W   & +15{,}39 Ah & +35{,}79 Ah & 50\% & PWM 20A & 14h \\
10  & 7{,}21  & 14{,}42 & 1 painel de 50W   & +13{,}59 Ah & +33{,}99 Ah & 50\% & PWM 20A & 14h \\
20  & 9{,}21  & 18{,}42 & 1 painel de 50W   & +11{,}59 Ah & +31{,}99 Ah & 50\% & PWM 20A & 14h \\
30  & 11{,}21 & 22{,}42 & 2 painéis de 50W  & +9{,}59 Ah  & +29{,}99 Ah & 50\% & PWM 20A & 14h \\
40  & 13{,}21 & 26{,}42 & 2 painéis de 50W  & +7{,}59 Ah  & +27{,}99 Ah & 50\% & PWM 20A & 14h \\
50  & 15{,}21 & 30{,}42 & 2 painéis de 50W  & +5{,}59 Ah  & +25{,}99 Ah & 50\% & PWM 20A & 14h \\
75  & 20{,}21 & 40{,}42 & 2 painéis de 50W  & \textcolor{red}{-1{,}41 Ah}  & +19{,}59 Ah & 50\% & PWM 20A & 14h \\
100 & 25{,}21 & 50{,}42 & 3 painéis de 50W* & \textcolor{red}{-6{,}41 Ah}  & +14{,}59 Ah & 50\% & PWM 20A & 14h \\
120 & 29{,}21 & 58{,}42 & 3 painéis de 50W* & \textcolor{red}{-10{,}41 Ah} & +10{,}59 Ah & 50\% & PWM 20A & 14h \\
150 & 34{,}21 & 68{,}42 & 4 painéis de 50W* & \textcolor{red}{-15{,}41 Ah} & +5{,}59 Ah  & 50\% & PWM 20A & 14h \\
180 & 39{,}21 & 78{,}42 & 4 painéis de 50W* & \textcolor{red}{-20{,}41 Ah} & +0{,}59 Ah  & 50\% & PWM 20A & 14h \\
\bottomrule
\end{tabularx}

\vspace{0.5em}
{\scriptsize\textit{Fonte: Elaborado pelo autor.}\par}
\end{table}

Considerações importantes sobre a tabela de dimensionado em capacidade de baterias e quantidade de placas solares:

\begin{itemize}
  \setlength\itemsep{6pt} % Espaço entre itens
  \setlength\parskip{0pt}
  \setlength\parsep{0pt}
  \item O controlador de carga PWM de 20A foi respeitado como constante e nunca ultrapassado em termos de 240W de Corrente de carga vinda dos painéis. Capacidade segura de operação com as baterias e painéis previstos.
  
  \item Todos os consumos dos componentes foram multiplicados pelas 14 horas de funcionamento diário, exceto os casos onde o tempo de acionamento é menor (como os relés e válvulas que funcionam apenas 10 minutos por dia).

  \item A bateria mínima indicada é o dobro do consumo diário (consumo ÷ 0,5), garantindo que você não descarregue mais do que 50\% da bateria para preservar a vida útil (Regra de segurança máxima).
  
  \item Número de painéis indicado para manter folga de energia e garantir autonomia, considerando 5h de sol pleno.

  \item Sobra de energia negativa (-) indica que o painel solar não é suficiente para suprir o consumo diário sozinho.

  \item Sobra positiva (+) indica energia extra disponível para recarga e eventual segurança.

  \item A "sobra de energia" mostra quanto ainda sobraria de energia solar por dia, considerando painéis de 50W.

  \item Situações com mais de 20 kits já recomendam o uso de 2 painéis de 50W para manter folga.
  
\end{itemize}

A execução do projeto exigirá recursos humanos como mão de obra especializada. A previsão orçamentária do projeto contempla:

\listarecuada{
    \item Desenvolvimento e suporte técnico (R\$ 2.277,00).
    \item Engenheiros eletrônicos e técnicos em suporte (R\$ 2.277,00).
    \item Diretoria Executiva, Estratégica e Financeiro (R\$ 2.277,00).
    \item Capacitação e marketing (R\$ 1.518,00).
}

O investimento estimado inicial para a elaboração do projeto é de aproximadamente R\$10.000,00. Levando em consideração o acréscimo de 500 reais para a compra de alguns materiais elétricos e de estrutura. Uma vez feito, o valor poderia ser passado para o consumidor final em formato de serviço de assinatura com um valor simbólico de R\$ 99,99 onde uma parcela mensal seria posta para que seja paga mensalmente (igual streaming de vídeos).

É relevante apontar que, com o passar do tempo, o projeto fique mais acessível à medida que novos agricultores entrem no programa. O orçamento mensal considera o custo com materiais, testes, eventuais substituições de componentes e um valor reservado para apoio técnico e capacitação de usuários durante o período de validação do sistema.

\vspace*{-1.0em}  % Ajuste
\section*{\fontsize{12pt}{14pt}\selectfont\textbf{2.4. O Sistema Atual}}
\addcontentsline{toc}{section}{2.4. O Sistema Atual} 

\needspace{5\baselineskip} % reserva espaço suficiente para a próxima seção
\vspace*{-1.0em}  % Ajuste
\section*{\fontsize{12pt}{14pt}\selectfont\textbf{2.4.1. Funcionamento do sistema atual}}
\addcontentsline{toc}{section}{2.4.1. Funcionamento do sistema atual} 
\vspace{-0.7\baselineskip}

O sistema atual é uma solução automatizada de monitoramento e controle de irrigação desenvolvida com base no microcontrolador ESP32 DevKit V1, integrando sensores ambientais e atuadores com conectividade à Arduino Cloud para visualização e controle remoto.

\selectfont\textbf{Componentes e funcionamento geral:}

\listarecuada{
    \item Microcontrolador principal: ESP32, que atua como o cérebro do sistema, executando a lógica de controle, comunicação e integração com a nuvem.
    \item Placa expansora: ajuda na utilização de fios no lugar de jumpers, para contatos mais confiáveis e comunicações não duvidosas entre circuitos.
}

\selectfont\textbf{Sensores conectados:}

\listarecuada{
    \item Sensores de umidade do solo: coletam dados do nível de umidade do solo.
    \item Sensor DHT11: mede temperatura e umidade do ar.
    \item Sensor LDR: detecta luminosidade ambiente.
}

\selectfont\textbf{Atuadores e controle:}

\listarecuada{
    \item Módulos relé: acionam válvulas solenoides de 12V responsáveis pela irrigação.
    \item Válvulas solenoides: controlam o fluxo de água para irrigação, ligadas via relés.
}

\selectfont\textbf{Expansor do número de pinos analógicos/digitais:}

\listarecuada{
    \item Multiplexador CD74HC4067: permite ler múltiplos sensores de umidade usando apenas 05 portas analógicas do ESP32, otimizando recursos do microcontrolador. Pode ser ligado no máximo 6 multiplexadores, resultando em 96 sensores totais em uma ESP32. (É recomendável usar resistores pull-down).
}

\selectfont\textbf{Interface e visualização:}

\listarecuada{
    \item Display LCD I2C 16x2: exibe informações locais como valores de sensores ou status do sistema.
    \item Comunicação Bluetooth: exibe informações contínuas das leituras por intermédio de aplicativo terminal para dispositivos seriais.
    \item Arduino Cloud: permite visualizar os dados coletados e acionar manualmente a irrigação de forma remota, por meio de dashboards online.
    \item Power BI e Google Sheets: Permite a análise de dados de forma mais detalhada, uma visão mais clara e de fácil compreensão dos dados.
}

\selectfont\textbf{Conectividade:}

\listarecuada{
    \item A ESP32 conecta-se à internet via Wi-Fi usando credenciais armazenadas em \textit{arduino\_secrets.h}.
    \item As variáveis e propriedades são sincronizadas com a Arduino Cloud conforme definidas em \textit{thingProperties.h}.
    \item A assistente virtual Alexa foi integrada por meio da plataforma Arduino Cloud com ajuda de uma Skills do próprio aplicativo Alexa.
}

\selectfont\textbf{Banco de dados:}

\listarecuada{
    \item Bibliotecas de integração: o uso de credenciais para armazenamentos no google sheets com ajuda de: \textit{WiFi.h}, \textit{HTTPClient.h} e \textit{ArduinoJson.h}.
}

\selectfont\textbf{Alimentação e energia:}

\listarecuada{
    \item Fontes de 12V e 5V alimentam sensores e módulos.
    \item Há previsão de uso de painéis solares e baterias seladas para tornar o sistema autônomo energeticamente. 
    \item Economia de energia: utilização de função especial do ESP32 para diminuir o consumo de energia para o mínimo com ajuda de: \textit{esp\_sleep\_enable\_timer\_wakeup()} e \textit{esp\_deep\_sleep\_start()}.
}

\selectfont\textbf{Operação:}

O projeto utiliza uma placa microcontroladora ESP32 DevKit V1 como cérebro central do sistema. Essa placa é responsável por controlar, ler e tomar decisões com base em dados coletados de diversos sensores ambientais, integrando automação, monitoramento e eficiência energética em um único sistema de irrigação inteligente.

O sistema lê ciclicamente os sensores utilizando a função millis() (Não Bloqueante), já que o uso da função delay() (Bloqueante), não é recomendado para projetos grandes por causa do travamento das outras execuções que devem continuar funcionando; para a leitura dos dados do solo, o sistema utiliza sensores de umidade posicionados em diferentes pontos da plantação ou jardim. Como o número de sensores necessários pode ser alto, o projeto emprega multiplexadores CD74HC4067, que permitem conectar até 16 sensores analógicos em cinco entrada da ESP32. Isso amplia significativamente a quantidade de sensores que podem ser lidos, otimizando o uso dos pinos disponíveis na placa. 

Além da umidade do solo, o sistema também realiza monitoramento climático local, por meio do sensor DHT11, que mede a temperatura e a umidade do ar, e do sensor de luminosidade LDR, que identifica a intensidade da luz ambiente. Com essas informações, o sistema consegue tomar decisões mais inteligentes sobre o momento ideal para irrigar, evitando, por exemplo, ativar as válvulas durante chuvas ou períodos muito úmidos, e claro evitar irrigação em momentos de alta luminosidade, devido ao horário quente, pode acabar sendo prejudicial a planta receber água em altas temperaturas.

A irrigação é feita por meio de válvulas solenoides 12V, que são ativadas ou desativadas por módulos relé controlados pela ESP32. Cada válvula pode ser conectada a um setor ou linha de irrigação diferente. O acionamento é automatizado com base nos níveis de umidade lidos, mas pode ser ajustado ou monitorado remotamente por meio de interações no Dashboard da Arduino Cloud ou comandos de voz via Alexa. 

A alimentação elétrica do projeto é bem dimensionada para a quantidade de sensores da preferência do cliente e para proteger a bateria para ser utilizada por muito tempo, garantindo autonomia e operação off-grid, sendo distribuído por fontes de alimentação de 5V e 12V, dependendo da infraestrutura disponível. O sistema é montado em caixas de proteção IP65, para garantir durabilidade e segurança em ambientes externos.

Para otimizar ainda mais o consumo de energia, o sistema utiliza a funcionalidade de sono profundo (deep sleep) da ESP32. Essa técnica permite que o microcontrolador entre em um estado de baixo consumo elétrico quando não está realizando leituras ou acionamentos, como durante a madrugada ou entre ciclos de irrigação. Supondo que o sistema permaneça em deep sleep por 10 horas contínuas por dia, o consumo de energia da placa pode cair de cerca de 160 mA (modo ativo) para menos de 10 µA (modo deep sleep). Isso representa uma redução de mais de 99\% no consumo elétrico da ESP32 durante esse período, resultando em significativa economia de energia e maior autonomia da bateria. Em um sistema alimentado por bateria e painel solar, essa estratégia é essencial para manter o funcionamento contínuo mesmo em dias nublados ou com baixa radiação solar, reduzindo a dependência de recargas frequentes e aumentando a confiabilidade da automação em campo. 

Essa integração torna o sistema ideal para aplicações em agricultura de precisão, hortas urbanas, jardins automatizados e estudos ambientais. 

\vspace*{-1.0em}  % Ajuste
\section*{\fontsize{12pt}{14pt}\selectfont\textbf{2.4.2. Problemas do sistema atual}}
\addcontentsline{toc}{section}{2.4.2. Problemas do sistema atual} 
\vspace{-0.7\baselineskip}

Apesar de o sistema atual representar um avanço no monitoramento e controle de irrigação, ele ainda apresenta algumas limitações e desafios que justificam a busca por melhorias e por um novo sistema mais robusto e confiável:

\vspace*{0.5em}  
\selectfont\textbf{Dependência de conexão Wi-Fi local:}
\vspace*{0.5em} 

O funcionamento do sistema depende diretamente da conectividade com redes Wi-Fi configuradas previamente. Em áreas rurais ou locais com instabilidade de sinal, isso pode comprometer a atualização de dados na nuvem e o controle remoto, além do envio dos dados para o banco. 

\vspace*{0.5em} 
\selectfont\textbf{Falta de tomada de decisão inteligente:}
\vspace*{0.5em} 

O sistema atual aciona as válvulas com base em valores de sensores pré-definidos (como umidade menor que certo valor). No entanto, não há algoritmos mais complexos de inteligência artificial, aprendizado ou lógica fuzzy para adaptar a irrigação a diferentes situações climáticas, tipos de solo ou culturas específicas. 

\vspace*{0.5em} 
\selectfont\textbf{Interface limitada para o usuário local:}
\vspace*{0.5em} 

Embora haja um display LCD 16x2, ele mostra uma quantidade limitada de informações. Em caso de falha na internet, o usuário local não consegue acessar históricos, diagnósticos ou gráficos detalhados diretamente pelo sistema.

\vspace*{0.5em} 
\selectfont\textbf{Pouca modularidade e escalabilidade:}
\vspace*{0.5em} 

A estrutura atual, mesmo com o uso do multiplexador, exige ajustes manuais no código para adicionar mais sensores ou válvulas. Isso limita a escalabilidade sem intervenções técnicas constantes. 

\vspace*{0.5em} 
\selectfont\textbf{Segurança e autenticação simplificadas:}
\vspace*{0.5em} 

As credenciais Wi-Fi e o acesso à nuvem são armazenados de forma básica, sem autenticação avançada ou criptografia de dados trafegados na hora de ir para o banco por protocolo HTTP, exceto os dados que vão para a Arduino cloud por meio de protocolo MQTT, o que pode ser um problema em termos de segurança da informação. 

\vspace*{0.5em} 
\selectfont\textbf{Manutenção e diagnóstico dificultados:}
\vspace*{0.5em} 

Não há um sistema de logs ou notificações locais/online para avisar falhas (como sensor desconectado, válvula que não responde ou queda de energia). Isso dificulta a manutenção preventiva e corretiva do sistema.

\vspace*{0.5em} 
\selectfont\textbf{Ausência de armazenamento local de dados:}
\vspace*{0.5em} 

Sem conexão com a internet, o sistema não registra os dados em memória local para posterior sincronização. Isso pode gerar perda de informações importantes em caso de falhas de rede.

\vspace*{0.5em} 
\selectfont\textbf{Uso de sensores baratos de baixa precisão:}
\vspace*{0.5em} 

O sistema usa sensores de umidade do solo de baixo custo, que são acessíveis, mas apresentam leituras instáveis e pouca durabilidade. São afetados por interferências elétricas, temperatura e oxidação, exigindo calibração frequente. Isso compromete a precisão dos dados e pode levar à irrigação incorreta.
