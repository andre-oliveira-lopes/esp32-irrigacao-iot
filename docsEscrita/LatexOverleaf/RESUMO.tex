\begin{center}
Irrigação Automatizada em Pomares e Pequenas Plantações: Uma Abordagem Baseada em Integração IoT na Plataforma Arduino Cloud \\[1.5em]
\end{center}

\begin{flushright}
\textbf{André Oliveira Lopes\textsuperscript{1}}
\end{flushright}

\noindent
\textbf{RESUMO:} A tecnologia IoT 5.0 vem se consolidando como ferramenta estratégica para modernizar o setor agrícola, essencial à economia brasileira. Este trabalho propõe um sistema automatizado de irrigação para pomares e pequenas plantações, visando reduzir deslocamentos, otimizar o tempo, minimizar riscos e custos operacionais. A solução utiliza a plataforma Arduino Cloud, sensores alimentados por energia solar e comunicação via protocolo MQTT, com o microcontrolador ESP32 DevKit v1 acoplado a uma placa expansora. Sensores de umidade do solo, temperatura e umidade do ar, luminosidade e outros fornecem dados processados por comandos em C++, que acionam relés e válvulas conforme a necessidade das plantas. As informações são exibidas em tempo real na nuvem do Arduino e armazenadas no Google Sheets, permitindo análise no Power BI. A proposta destaca-se pela acessibilidade e flexibilidade, podendo ser integrada a assistentes virtuais como a Alexa e adaptada a diferentes perfis de produtores. Conclui-se que o uso da IoT na irrigação oferece mais eficiência, economia, sustentabilidade e segurança ao campo. 

\vskip\baselineskip

\noindent
\textbf{Palavras-chave}: IoT 5.0; irrigação; Arduino Cloud; Placas solares; Microcontrolador.
\vskip\baselineskip

\noindent
\textbf{ABSTRACT:} IoT 5.0 technology has been consolidating itself as a strategic tool to modernize the agricultural sector, which is essential to the Brazilian economy. This work proposes an automated irrigation system for orchards and small plantations, aiming to reduce travel, optimize time, minimize risks and operational costs. The solution uses the Arduino Cloud platform, solar-powered sensors and communication via the MQTT protocol, with the ESP32 DevKit v1 microcontroller coupled to an expansion board. Soil moisture, air temperature and humidity, light and other sensors provide data processed by C++ commands, which activate relays and valves according to the plants' needs. The information is displayed in real time in the Arduino cloud and stored in Google Sheets, allowing analysis in Power BI. The proposal stands out for its accessibility and flexibility, and can be integrated with virtual assistants such as Alexa and adapted to different producer profiles. It is concluded that the use of IoT in irrigation offers greater efficiency, savings, sustainability and safety to the field.

\vskip\baselineskip

\noindent
\textbf{KEYWORDS}: IoT 5.0; irrigation; Arduino Cloud; solar panels; microcontroller.

\noindent
\footnotetext[1]{Graduando em Ciência da Computação — Faculdade Estácio Teresina. E-mail:\\ andreoliveiralopes123@gmail.com}
