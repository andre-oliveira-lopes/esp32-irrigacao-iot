% Fonte: Tam 12 pt
\documentclass[12pt,a4paper]{article}  

% ---------------------------- INÍCIO DAS CONFIGURAÇÕES ----------------------
% Usa Arial real
\usepackage{fontspec}
\setmainfont{Arial}

% Idioma e formatação
\usepackage[brazil]{babel}  % Hifenização e idioma
\usepackage[none]{hyphenat} % Isso impede o LaTeX de fazer qualquer quebra silábica
\usepackage{graphicx}       % Usar Imagens
\usepackage{float} % para usar [H] e fixar a imagem
\usepackage{indentfirst}    % Faz o primeiro parágrafo de cada seção também tenha recuo.
\usepackage{setspace}

% Espaçamento 1.5 e margens ABNT
\usepackage[a4paper,top=3cm,bottom=2cm,left=3cm,right=2cm]{geometry}
\onehalfspacing
\setlength{\parindent}{1.25cm}  % ABNT: recuo de parágrafo
\setlength{\parskip}{0pt}         % Espaço entre parágrafos (ABNT: não deve haver)

% hang: alinha o número da nota à esquerda e o texto pendente e flushmargin: remove o recuo da primeira linha da nota de rodapé.
\usepackage[hang,flushmargin]{footmisc}

% referencias bibliograficas
\usepackage{url}
\usepackage{ragged2e}
\usepackage{changepage}
\usepackage{hyperref}       % Links clicáveis
\hypersetup{
  colorlinks=false,
  pdfborder={0 0 0},
  hidelinks
}
\urlstyle{same}   % URLs com mesma fonte do texto
\sloppy           % Permite quebras de linha em URLs longas

% para utilizar o sumario
\usepackage{tocloft}

\renewcommand{\contentsname}{} % evita erro posterior
\renewcommand{\cfttoctitlefont}{\hspace*{-1000pt}} % empurra o título para fora
\renewcommand{\cftaftertoctitle}{}
\setlength{\cftbeforetoctitleskip}{-3.4em}
\setlength{\cftaftertoctitleskip}{0.5em}

\renewcommand{\cftdotsep}{1} % Tamanho da separação entre os pontinhos. 
\renewcommand{\cftsecleader}{\cftdotfill{\cftdotsep}}   % se tiver seções
\renewcommand{\cftsubsecleader}{\cftdotfill{\cftdotsep}}  % se tiver subseções

\renewcommand{\cftsecfont}{\normalfont}       % Remove negrito dos títulos
\renewcommand{\cftsubsecfont}{\normalfont}    % Remove negrito de subtítulos
\renewcommand{\cftsecpagefont}{\normalfont}   % Números também sem negrito
\renewcommand{\cftsubsecpagefont}{\normalfont}

\setlength{\cftbeforesecskip}{0.5em}          % Espaçamento entre seções
\setlength{\cftsecindent}{0pt}                % Sem indentação

\hbadness=10000 % Isso desativa os avisos de Underfull \hbox com badness abaixo de 10000

% paginção das paginas 
\usepackage{fancyhdr}    % Controle total do cabeçalho e rodapé
\setlength{\headheight}{15pt} % cabeçalho e rodapé 
\usepackage{lastpage}    % (opcional) se quiser contar total de páginas
\pagestyle{fancy}        % Ativa o estilo personalizado
\fancyhf{}               % Limpa cabeçalho e rodapé
\renewcommand{\headrulewidth}{0pt} % Remove linha de topo
\fancyhead[R]{\thepage} % Posiciona o número no canto superior direito

% imentando textos
\usepackage{needspace}

% fazer tabela
\usepackage{array}
\usepackage{booktabs}
\usepackage{tabularx}
\usepackage{caption}
\usepackage{color, colortbl, xcolor} % Para \textcolor
\usepackage{longtable}
\usepackage{listings}
% ----------------- para uso do JavaScript
\lstdefinelanguage{JavaScript}{
  keywords={function, var, return, if, else, try, catch, new},
  keywordstyle=\color{blue}\bfseries,
  ndkeywords={class, export, boolean, throw},
  ndkeywordstyle=\color{gray}\bfseries,
  identifierstyle=\color{black},
  sensitive=false,
  comment=[l]{//},
  morecomment=[s]{/*}{*/},
  commentstyle=\color{gray}\ttfamily,
  stringstyle=\color{red}\ttfamily,
  morestring=[b]',
  morestring=[b]"
}

\lstdefinestyle{meucodigo}{
  language=JavaScript,
  basicstyle=\ttfamily\scriptsize,
  breaklines=true,
  breakatwhitespace=false,
  keepspaces=true,
  columns=fullflexible,
  frame=none,
  tabsize=2,
  showstringspaces=false
}
% ------------- para uso da Linguagem M
\lstdefinelanguage{PowerQuery}{
  keywords={let, in, each, if, then, else, true, false, null},
  morekeywords={[2]Table, Record, List, Text, Number, Function, Value},
  keywordstyle=\color{blue}\bfseries,
  keywordstyle=[2]\color{teal},
  identifierstyle=\color{black},
  sensitive=true,
  comment=[l]//,
  commentstyle=\color{gray}\ttfamily,
  stringstyle=\color{red}\ttfamily,
  morestring=[b]",
  morestring=[b]'
}

\lstdefinestyle{meuEstiloM}{
  language=PowerQuery,
  basicstyle=\ttfamily\scriptsize,
  breaklines=true,
  columns=fullflexible,
  keepspaces=true,
  showstringspaces=false,
  frame=none,
  tabsize=2
}

% ---------------------------- FIM DAS CONFIGURAÇÕES ---------------------- 

% ------------- FUNÇÃO lista recuada CRIADA PARA SER CHAMADA ------------
\newcommand{\listarecuada}[1]{%
  \begingroup
  \vspace*{1.0em}  % Ajuste
  \setstretch{1.0}%
  \fontsize{10pt}{12pt}\selectfont
  \begin{adjustwidth}{4cm}{0cm}%
  \begin{enumerate}
  #1
  \end{enumerate}
  \end{adjustwidth}
  \endgroup
}
% ------------- Fim da FUNÇÃO lista recuada------------
% ------------- FUNÇÃO imagem com fonte CRIADA PARA SER CHAMADA ------------
\newcommand{\minhaimagemcomfonte}[3]{%
  \begin{figure}[H]
  \begin{center}
  \caption{\textbf{#3}}
  \includegraphics[width=#1]{#2}
  \vspace{0.5em}
  
  {\footnotesize\textit{Fonte: Elaborado pelo autor.}\par}
  \end{center}
  \end{figure}
}
% ------------- Fim da FUNÇÃO imagem com fonte------------

\begin{document}

\begin{titlepage}
\begin{center}

\includegraphics[width=8cm]{Logo_Estacio}

\textbf{FACULDADE ESTÁCIO TERESINA}\\[0.5em]
\textbf{CURSO BACHARELADO EM CIÊNCIAS DA COMPUTAÇÃO}\\[4em]

\textbf{ANDRÉ OLIVEIRA LOPES}\\[12em]

\textbf{IRRIGAÇÃO AUTOMATIZADA EM POMARES E PEQUENAS PLANTAÇÕES:}\\
\textbf{UMA ABORDAGEM BASEADA EM INTEGRAÇÃO IOT NA PLATAFORMA ARDUINO CLOUD}

\vfill

\textbf{Teresina - PI}\\
\textbf{2025}

\end{center}
\end{titlepage}

\pagenumbering{gobble} % Esconde numero de pagina
\begin{titlepage}
\begin{center}

\textbf{FACULDADE ESTÁCIO TERESINA}\\[4em]

\textbf{ANDRÉ OLIVEIRA LOPES}\\[14em]

\textbf{IRRIGAÇÃO AUTOMATIZADA EM POMARES E PEQUENAS PLANTAÇÕES: \\UMA ABORDAGEM BASEADA EM INTEGRAÇÃO IOT NA PLATAFORMA ARDUINO CLOUD}\\[6em]

\begin{adjustwidth}{4cm}{0cm}
\justifying
Trabalho apresentado ao curso de Ciência da Computação da Faculdade Estácio Teresina, como requisito parcial para a obtenção do grau de Bacharel em Ciência da Computação.
\end{adjustwidth}

\begin{adjustwidth}{4cm}{0cm}
\vspace{1.5em}
Orientadora: Maria Guimarães
\end{adjustwidth}

\vfill

\textbf{TERESINA – PI\\}
\textbf{2025}

\end{center}
\end{titlepage}


\begin{center}
\textbf{SUMÁRIO}
\end{center}
\tableofcontents

\begin{center}
Irrigação Automatizada em Pomares e Pequenas Plantações: Uma Abordagem Baseada em Integração IoT na Plataforma Arduino Cloud \\[1.5em]
\end{center}

\begin{flushright}
\textbf{André Oliveira Lopes\textsuperscript{1}}
\end{flushright}

\noindent
\textbf{RESUMO:} A tecnologia IoT 5.0 vem se consolidando como ferramenta estratégica para modernizar o setor agrícola, essencial à economia brasileira. Este trabalho propõe um sistema automatizado de irrigação para pomares e pequenas plantações, visando reduzir deslocamentos, otimizar o tempo, minimizar riscos e custos operacionais. A solução utiliza a plataforma Arduino Cloud, sensores alimentados por energia solar e comunicação via protocolo MQTT, com o microcontrolador ESP32 DevKit v1 acoplado a uma placa expansora. Sensores de umidade do solo, temperatura e umidade do ar, luminosidade e outros fornecem dados processados por comandos em C++, que acionam relés e válvulas conforme a necessidade das plantas. As informações são exibidas em tempo real na nuvem do Arduino e armazenadas no Google Sheets, permitindo análise no Power BI. A proposta destaca-se pela acessibilidade e flexibilidade, podendo ser integrada a assistentes virtuais como a Alexa e adaptada a diferentes perfis de produtores. Conclui-se que o uso da IoT na irrigação oferece mais eficiência, economia, sustentabilidade e segurança ao campo. 

\vskip\baselineskip

\noindent
\textbf{Palavras-chave}: IoT 5.0; irrigação; Arduino Cloud; Placas solares; Microcontrolador.
\vskip\baselineskip

\noindent
\textbf{ABSTRACT:} IoT 5.0 technology has been consolidating itself as a strategic tool to modernize the agricultural sector, which is essential to the Brazilian economy. This work proposes an automated irrigation system for orchards and small plantations, aiming to reduce travel, optimize time, minimize risks and operational costs. The solution uses the Arduino Cloud platform, solar-powered sensors and communication via the MQTT protocol, with the ESP32 DevKit v1 microcontroller coupled to an expansion board. Soil moisture, air temperature and humidity, light and other sensors provide data processed by C++ commands, which activate relays and valves according to the plants' needs. The information is displayed in real time in the Arduino cloud and stored in Google Sheets, allowing analysis in Power BI. The proposal stands out for its accessibility and flexibility, and can be integrated with virtual assistants such as Alexa and adapted to different producer profiles. It is concluded that the use of IoT in irrigation offers greater efficiency, savings, sustainability and safety to the field.

\vskip\baselineskip

\noindent
\textbf{KEYWORDS}: IoT 5.0; irrigation; Arduino Cloud; solar panels; microcontroller.

\noindent
\footnotetext[1]{Graduando em Ciência da Computação — Faculdade Estácio Teresina. E-mail:\\ andreoliveiralopes123@gmail.com}


\clearpage  % Esconde numero de pagina
\pagenumbering{arabic} % Esconde numero de pagina
\setcounter{page}{6} % A introdução inicia com esse numero

% --- textbf = mantem o texto em negrito. 
\section*{\fontsize{12pt}{14pt}\selectfont\textbf{INTRODUÇÃO}} % 
\addcontentsline{toc}{section}{INTRODUÇÃO}  % colocar no sumário
\vspace*{-0.7\baselineskip}  % Remove o espaço extra entre título e texto

A agricultura familiar desempenha um papel fundamental na produção de alimentos no Brasil, mas ainda enfrenta grandes desafios relacionados ao uso eficiente da água, à dependência de mão de obra e à ausência de tecnologias acessíveis. Em pequenas propriedades e pomares, a irrigação é frequentemente realizada de forma manual e imprecisa, o que resulta em desperdício de recursos, baixa produtividade e esforço físico elevado por parte dos agricultores. Nesse cenário, a integração de tecnologias de Internet das Coisas (IoT) apresenta-se como uma alternativa viável e promissora para automatizar processos, otimizar o uso da água e melhorar a gestão hídrica nas lavouras.

Este trabalho propõe o desenvolvimento de um sistema automatizado de irrigação, voltado especialmente para pequenos e médios cultivos, utilizando o microcontrolador ESP32 DevKit V1 em conjunto com sensores ambientais, multiplexadores, relés e válvulas solenoides. O sistema permite o monitoramento em tempo real da umidade do solo, luminosidade, temperatura e umidade do ar, com tomada de decisão automática baseada em dados sensoriais. A comunicação com a plataforma Arduino IoT Cloud possibilita o controle remoto das funções por meio de dashboards online, bem como a integração com assistentes virtuais como a Alexa.

Além de funcionalidades como o modo de economia de energia (deep sleep) e a previsão de alimentação por energia solar, o sistema também conta com mecanismos de envio de dados para o Google Sheets e integração com o Power BI para análise e visualização. A solução se destaca por ser acessível, escalável e de fácil adaptação, sendo projetada para operar em regiões com infraestrutura limitada. Ao aliar automação, baixo custo e sustentabilidade, o projeto visa promover uma agricultura mais eficiente, tecnológica e voltada à realidade dos pequenos produtores.
% CAPÍTULO 1 – PROBLEMA

\vspace*{-1.0em} % Ajuste conforme visual pra iniciar "1 PROBLEMA". 
\section*{\fontsize{12pt}{14pt}\selectfont\textbf{1. PROBLEMA}}
\addcontentsline{toc}{section}{1. PROBLEMA}
\vspace{-0.7\baselineskip}

\section*{\fontsize{12pt}{14pt}\selectfont\textbf{1.1. Tema do Trabalho}}
\addcontentsline{toc}{section}{1.1. Tema do Trabalho}
\vspace{-0.7\baselineskip}

A situação-problema abordada refere-se à irrigação, uma técnica essencial para o suprimento hídrico das plantas, que, quando realizada por métodos tradicionais, pode resultar em uso ineficiente da água, desperdícios e escassez desse recurso vital, por isso o presente tema escolhido foi; Irrigação Automatizada em Pomares e Pequenas Plantações: Uma Abordagem Baseada em Integração IoT na Plataforma Arduino Cloud.

\vspace*{-1.0em} % Ajuste 
\section*{\fontsize{12pt}{14pt}\selectfont\textbf{1.2. Contextualização}}
\addcontentsline{toc}{section}{1.2. Contextualização}
\vspace{-0.7\baselineskip}

A agricultura familiar representa uma parcela significativa da produção agrícola brasileira, especialmente em regiões onde a mecanização ainda é limitada. Em pequenas propriedades e pomares, o manejo hídrico costuma ser realizado de forma manual, exigindo longas jornadas de trabalho e gerando, muitas vezes, desperdício de água.

Com o avanço das tecnologias de Internet das Coisas (IoT), torna-se viável implementar soluções automatizadas e acessíveis que otimizem recursos, aumentem a produtividade e melhorem a qualidade de vida dos agricultores. O sistema proposto destina-se, portanto, a pequenos produtores rurais, com foco especial em propriedades que não possuem acesso a tecnologias caras ou infraestrutura avançada.

\vspace*{-1.0em} % Ajuste 
\section*{\fontsize{12pt}{14pt}\selectfont\textbf{1.3. Situação-Problema}}
\addcontentsline{toc}{section}{1.3. Situação-Problema}
\vspace{-0.7\baselineskip}

A irrigação manual e imprecisa nas pequenas propriedades leva ao uso ineficiente da água, impactando diretamente a produtividade e a sustentabilidade da lavoura. Além disso, o esforço físico constante exigido para essa tarefa sobrecarrega os trabalhadores, especialmente em famílias com pouca mão de obra disponível. A ausência de controle em tempo real e de automação impede decisões rápidas e assertivas, comprometendo o rendimento das colheitas e o bem-estar dos produtores.

\vspace*{-1.0em} % Ajuste 
\section*{\fontsize{12pt}{14pt}\selectfont\textbf{1.4. Breve Descrição da Solução}}
\addcontentsline{toc}{section}{1.4. Breve Descrição da Solução}
\vspace{-0.7\baselineskip}

Foi desenvolvido um sistema automatizado de irrigação baseado em tecnologia IoT, utilizando sensores ambientais integrados à plataforma Arduino Cloud devido à compatibilidade com o protocolo MQTT e à placa ESP32 DevKit V1. A solução permite o desenvolvimento de um dashboard para visualização em tempo real de umidade do solo, luminosidade, temperatura e umidade do ar e outros, acionando automaticamente válvulas solenoides conforme a necessidade.

Os dados são armazenados no Google Sheets e podem ser analisados posteriormente via Power BI. A integração com a assistente virtual Alexa amplia a acessibilidade, facilitando o uso do sistema mesmo por pessoas com pouco domínio tecnológico. O projeto visa aumentar a autonomia, reduzir o desperdício de água e melhorar a qualidade de vida do agricultor familiar.
% CAPÍTULO 2 – DESENVOLVIMENTO

\vspace*{-1.0em}  % Ajuste 
\section*{\fontsize{12pt}{14pt}\selectfont\textbf{2. CONTEXTUALIZAÇÃO DO PROBLEMA}}
\addcontentsline{toc}{section}{2 CONTEXTUALIZAÇÃO DO PROBLEMA}
\vspace{-0.7\baselineskip}

A irrigação é uma técnica essencial para o cultivo de diversas culturas, especialmente em regiões sujeitas a estiagens prolongadas. Entretanto, métodos tradicionais de irrigação, como o uso manual de mangueiras ou baldes, ainda são amplamente utilizados por agricultores familiares, o que resulta em desperdício de água, baixa eficiência e alta dependência de mão de obra. Tais práticas comprometem a produtividade e a sustentabilidade das lavouras.

Com a crescente escassez de recursos hídricos e os desafios enfrentados no campo, a automação do processo de irrigação, por meio da integração com tecnologias de Internet das Coisas (IoT), surge como uma solução promissora. Essa abordagem permite não apenas a economia de água, mas também a otimização do tempo de trabalho e a redução do esforço físico exigido dos agricultores.

\needspace{5\baselineskip} % reserva espaço suficiente para a próxima seção
\vspace*{-1.0em}  % Ajuste
\section*{\fontsize{12pt}{14pt}\selectfont\textbf{2.1. Premissas e Restrições do Projeto}}
\addcontentsline{toc}{section}{2.1 Premissas e Restrições do Projeto}
\vspace{-0.7\baselineskip}

Para a viabilização do sistema proposto, algumas premissas foram estabelecidas:

\listarecuada{
    \item Acesso estável à internet na área de operação.
    \item Disponibilidade de energia elétrica, ou fontes alternativas como placas solares.
    \item Espaço físico adequado para testes e instalação de sensores ambientais.
    \item Precisão satisfatória de sensores de baixo custo.
}

Contudo, o projeto também enfrenta algumas restrições importantes:

\listarecuada{
  \setcounter{enumi}{4}  % Começa do item 5
  \item Orçamento limitado, o que restringe o uso de equipamentos sofisticados.
  \item Manutenção frequente dos sensores, exigindo suporte contínuo.
  \item Calibração dos sensores que pode ser complexa e dependente de ajustes manuais no código.
  \item Instabilidades de conectividade em áreas rurais com sinal fraco.
  \item Integração dos módulos IoT, que demandou diversas adaptações técnicas.
  \item Gestão do consumo energético, especialmente em cenários com adoção de fontes renováveis.
}

Portanto, esses são as premissas e algumas restrições do projeto.

\needspace{5\baselineskip} % reserva espaço suficiente para a próxima seção
\vspace*{-1.0em}  % Ajuste
\section*{\fontsize{12pt}{14pt}\selectfont\textbf{2.2. Caracterização da Empresa}}
\addcontentsline{toc}{section}{2.2 Caracterização da Empresa}

\needspace{5\baselineskip} % reserva espaço suficiente para a próxima seção
\vspace*{-1.0em}  % Ajuste
\section*{\fontsize{12pt}{14pt}\selectfont\textbf{2.2.1. Histórico da Empresa}}
\addcontentsline{toc}{section}{2.2.1. Histórico da Empresa}
\vspace{-0.7\baselineskip}

A AgroIoT Brasil é uma startup fictícia criada em 2025, nascida do ambiente acadêmico, com o propósito de democratizar o acesso à automação agrícola entre pequenos e médios produtores. A iniciativa surgiu da observação das dificuldades enfrentadas por agricultores familiares, principalmente no que diz respeito ao manejo manual da irrigação e à falta de tecnologias acessíveis no campo. 

A empresa iniciou suas atividades com uma equipe reduzida, composta por profissionais das áreas de engenharia, agronomia e tecnologia da informação, atuando em regime de colaboração.

\needspace{5\baselineskip} % reserva espaço suficiente para a próxima seção
\vspace*{-1.0em}  % Ajuste
\section*{\fontsize{12pt}{14pt}\selectfont\textbf{2.2.2. Atividades da Empresa}}
\addcontentsline{toc}{section}{2.2.2. Atividades da Empresa}
\vspace{-0.7\baselineskip}

A AgroIoT Brasil dedica-se à pesquisa, desenvolvimento e comercialização de kits de irrigação automatizada com monitoramento remoto, compostos por:

\listarecuada{
    \item Sensores de umidade do solo, luminosidade, temperatura e umidade do ar e outros.
    \item Módulos de conectividade (ESP32 + Arduino Cloud via MQTT).
    \item Dashboards para visualização de dados antigos e em tempo real, além de transmissão continua via bluetooth.
    \item Integração com assistente virtual Alexa.
    \item Setor Administrativo e Financeiro.
}

Além da oferta dos kits, a empresa também fornece serviços como instalação, suporte técnico e treinamento prático para os usuários.

\needspace{5\baselineskip} % reserva espaço suficiente para a próxima seção
\vspace*{-1.0em}  % Ajuste
\section*{\fontsize{12pt}{14pt}\selectfont\textbf{2.2.3. Mercado Consumidor}}
\addcontentsline{toc}{section}{2.2.3. Mercado Consumidor}
\vspace{-0.7\baselineskip}

O público-alvo da AgroIoT Brasil inclui:

\listarecuada{
    \item Agricultores familiares.
    \item Horticultores urbanos.
    \item Cooperativas agroecológicas.
    \item Proprietários de pequenas e médias propriedades rurais.
    \item Instituições de ensino técnico com foco em práticas agrícolas.
}
Esses consumidores compartilham características como limitação de recursos financeiros, escassez de mão de obra e interesse crescente por soluções tecnológicas de fácil adoção.

\needspace{5\baselineskip} % reserva espaço suficiente para a próxima seção
\vspace*{-1.0em}  % Ajuste
\section*{\fontsize{12pt}{14pt}\selectfont\textbf{2.2.4. Concorrência}}
\addcontentsline{toc}{section}{2.2.4. Concorrência}
\vspace{-0.7\baselineskip}

No mercado brasileiro, empresas como Netafim e Toro dominam o setor de irrigação de precisão com soluções voltadas para grandes lavouras e projetos agrícolas de grande escala. Em contraste, a AgroIoT Brasil foca em atender um nicho ainda pouco explorado: o pequeno e médio produtor rural. Sua proposta de valor reside em: 

\listarecuada{
    \item Soluções personalizadas e de baixo custo.
    \item Suporte técnico acessível.
    \item Capacitação com linguagem didática e aplicação prática.
}

\needspace{5\baselineskip} % reserva espaço suficiente para a próxima seção
\vspace*{-1.0em}  % Ajuste
\section*{\fontsize{12pt}{14pt}\selectfont\textbf{2.2.5. Organograma}}
\addcontentsline{toc}{section}{2.2.5. Organograma} 
\vspace{-0.7\baselineskip}

A estrutura organizacional da AgroIoT Brasil é enxuta e baseada em metodologias ágeis. A empresa se divide em cinco núcleos principais:

\listarecuada{
    \item Diretoria Executiva e Estratégica – responsável pela tomada de decisões e definição de metas.
    \item Departamento de Engenharia e Desenvolvimento – voltado à criação e teste de novas soluções.
    \item Setor Comercial e Marketing – encarregado da divulgação e relacionamento com clientes.
    \item Atendimento ao Cliente e Suporte Técnico – apoio contínuo aos usuários.
    \item Setor Administrativo e Financeiro – gestão de recursos e planejamento financeiro.
}

A estrutura colaborativa permite flexibilidade, inovação contínua e forte aproximação com o público atendido.

\needspace{5\baselineskip} % reserva espaço suficiente para a próxima seção
\vspace*{-1.0em}  % Ajuste
\section*{\fontsize{12pt}{14pt}\selectfont\textbf{2.3. Proposta de Trabalho}}
\addcontentsline{toc}{section}{2.3. Proposta de Trabalho} 

\needspace{5\baselineskip} % reserva espaço suficiente para a próxima seção
\vspace*{-1.0em}  % Ajuste
\section*{\fontsize{12pt}{14pt}\selectfont\textbf{2.3.1. Método do Trabalho}}
\addcontentsline{toc}{section}{2.3.1. Método do Trabalho} 
\vspace{-0.7\baselineskip}

O desenvolvimento do projeto seguirá uma abordagem de pesquisa aplicada com ênfase em metodologia experimental, utilizando o modelo incremental com práticas de desenvolvimento ágil (Scrum). Isso permitirá validações contínuas por meio de ciclos curtos de desenvolvimento, testes práticos e refinamento da solução com base no feedback obtido em campo. 

\selectfont\textbf{Levantamento de dados:}

Os dados necessários para o desenvolvimento do sistema foram (ou serão) obtidos por meio de:

\listarecuada{
    \item Entrevistas com agricultores familiares, para entender as principais dificuldades enfrentadas com a irrigação manual.
    \item Observação direta em pequenas propriedades rurais, analisando padrões de irrigação, tipo de cultivo e uso atual de tecnologias.
    \item Pesquisa bibliográfica em fontes acadêmicas, artigos técnicos e manuais sobre sensores de umidade, automação agrícola e sistemas IoT.
}

\selectfont\textbf{Metodologia de desenvolvimento:}

Será utilizado o modelo incremental com base em Scrum, permitindo:

\listarecuada{
    \item Divisão do projeto em sprints quinzenais.
    \item Autoavaliação de execução individual.
    \item Protótipos evolutivos com incrementos funcionais a cada ciclo.
}

\selectfont\textbf{Ferramentas e técnicas de modelagem:}

\listarecuada{
    \item Diagramas de caso de uso, sequência e classes (UML) para modelagem dos requisitos e funcionalidades.
    \item Arduino IDE para desenvolvimento do código embarcado.
    \item Arduino Cloud para interface com a nuvem e dashboards de visualização.
    \item Fritzing para o protótipo esquemático do circuito eletrônico (Anexo II).
    \item TinkerCAD com YouTube para simulações iniciais do projeto.
    \item Comandos via Alexa para a acessibilidade das funções.
}

\needspace{5\baselineskip} % reserva espaço suficiente para a próxima seção
\vspace*{-1.0em}  % Ajuste
\section*{\fontsize{12pt}{14pt}\selectfont\textbf{2.3.2. Previsão e Alocação de Recursos (Humanos e Materiais)}}
\addcontentsline{toc}{section}{2.3.2. Previsão e Alocação de Recursos (Humanos e Materiais)} 
\vspace{-0.7\baselineskip}

\selectfont\textbf{Recursos humanos:}

\listarecuada{
    \item Desenvolvedor Full Stack / Maker (responsável por programação, testes, integração com nuvem).
    \item Engenheiro eletrônico ou técnico em eletrônica (auxílio com sensores, circuitos e calibração).
    \item Especialista de campo (auxílio na aplicação em propriedade rural e coleta de feedback).
    \item Coordenador do projeto (o próprio autor, responsável pelo planejamento, cronograma e documentação).
    \item Apoio eventual de professores orientadores e colegas da área agrícola.
}

\selectfont\textbf{Recursos materiais:}

\listarecuada{
    \item Placas ESP32 DevKit V1.  
    \item Esp32 Base Board.
    \item Sensores de umidade do solo.
    \item Sensores de Temperatura e umidade do ar (DHT11).
    \item Luminosidade (LDR).
    \item Módulos relé de 5v.
    \item Válvulas solenoides 12V.
    \item Multiplexador CD74HC4067.
    \item Painéis solares 50W / 12V + Reguladores de carga de 20A a 30A. 
    \item Bateria selada ou Li-Ion de 8A. 
    \item Fontes de alimentação de 12v e 5v para os sensores.
    \item Display LCD I2C 16x2. 
    \item Conectores de Derivação Emenda Wago: Modelo 221 ou 222.
    \item Fios bipolar paralelo cristal de 1,5mm² ou 2mm²
    \item Caso compre um relé com transistor ligado de forma diferente, você vai precisar de: 
    \begin{enumerate}
        \item Outro transistor, ele é o BC547.
        \item Um resistor de 10k Ohms.
        \item Placa De Circuito Perfurada Fenolite (O tamanho depende da necessidade).  
    \end{enumerate}
    \item Caixas de proteção IP65 para uso externo.
    \item Canos de 20 ou 25 mm com conectores (dependendo da necessidade).
    \item Ferramentas de desenvolvimento (gratuitas ou open-source).
    \item Conta ativa na plataforma Arduino Cloud.
    \item Ferramentas como: Alicate de ponta e de corte, chave de fenda e estrela, multímetro, chave de teste elétrico, filtro de linhas com várias tomadas para testes, estilete, cabo extensor de USB de qualidade, são opcionais, mas ajudaram no decorrer da construção.
}

\vspace*{-1.0em}  % Ajuste
\section*{\fontsize{12pt}{14pt}\selectfont\textbf{2.3.3. Cronograma de Trabalho (Diagrama de Gantt)}}
\addcontentsline{toc}{section}{2.3.3. Cronograma de Trabalho (Diagrama de Gantt)} 
\vspace{-0.7\baselineskip}

O projeto foi planejado para ocorrer em 6 meses, conforme abaixo:

\minhaimagemcomfonte{12cm}{capitulos/img/2_3_3_Cronograma_gantt}{Cronograma de Execução do Projeto}

\needspace{5\baselineskip} % reserva espaço suficiente para a próxima seção
\vspace*{-1.0em}  % Ajuste
\section*{\fontsize{12pt}{14pt}\selectfont\textbf{2.3.4. Previsão Orçamentária}}
\addcontentsline{toc}{section}{2.3.4. Previsão Orçamentária} 
\vspace{-0.7\baselineskip}

Com base nos recursos materiais de automação mencionados, estima-se um investimento mínimo de R\$ 1.160,00, distribuído da seguinte forma:

\begin{table}[H]
\centering
\captionsetup{justification=centering}
\caption{\textbf{Distribuição estimada de custos dos materiais do projeto}}
\vspace{-0.5em}
\footnotesize
\renewcommand{\arraystretch}{1.2} % espaçamento entre linhas

\begin{tabularx}{\textwidth}{>{\raggedright\arraybackslash}X c c}
\toprule
\textbf{ITEM} & \textbf{UNI.} & \textbf{PREÇO MÉDIO (BRL)} \\
\midrule
Placa ESP32 com Wi-Fi, Bluetooth ESP32S IDE Dual Core - Dev Kit V1 + Cabo Micro USB & 1 & R\$ 90,00 \\
Placa de Expansão para ESP32-DevKit V1 30 Pinos (Base Board) & 1 & R\$ 133,00 \\
Sensores de umidade do solo & 1 ou mais & R\$ 80,00 \\
Sensores de temperatura e umidade do ar (DHT11) & 1 & R\$ 22,00 \\
Módulo Sensor de Luminosidade Fotoresistor LDR & 1 & R\$ 10,00 \\
Módulos relé 1 canal de 5V & 1 ou mais & R\$ 14,00 \\
Válvulas solenoides 12V & 1 ou mais & R\$ 55,00 \\
Multiplexador CD74HC4067 & 1 & R\$ 16,00 \\
Painéis solares 50W / 12V + Regulador de carga de 20A & 1 & R\$ 120,00 \\
Painéis solares 50W / 12V (adicional) & 1 & R\$ 70,00 \\
Bateria selada 12V / 20Ah & 1 & R\$ 300,00 \\
Fonte de alimentação de 12V 2A para sensores & 1 & R\$ 40,00 \\
Fonte de alimentação de 5V 2A para sensores & 1 & R\$ 30,00 \\
Display LCD I2C 16x2 & 1 & R\$ 40,00 \\
Caixas de proteção IP65 para uso externo & 1 & R\$ 100,00 \\
Conta ativa na plataforma Arduino Cloud & 1 & R\$ 40,00 \\
Ferramentas de desenvolvimento (open-source) & 1 & Gratuito \\
\bottomrule
\end{tabularx}

\vspace{0.5em}
{\footnotesize\textit{Fonte: Elaborado pelo autor.}\par}
\end{table}

Com base nos recursos materiais elétricos e de estrutura para a construção do projeto em ambiente rural (Não incluso no valor acima por ser algo escalável):

\begin{table}[H]
\centering
\captionsetup{justification=centering}
\caption{\textbf{Materiais adicionais para montagem e instalação do sistema}}
\vspace{-0.5em}
\footnotesize
\renewcommand{\arraystretch}{1.2}

\begin{tabularx}{\textwidth}{>{\raggedright\arraybackslash}X c c}
\toprule
\textbf{ITEM} & \textbf{UNI.} & \textbf{PREÇO MÉDIO (BRL)} \\
\midrule
Canos de 20 ou 25 mm (PVC) & 1 ou mais & R\$ 20,00 à 40,00 \\
Conectores de Derivação Emenda Wago: 221 ou 222 & 1 ou mais & R\$ 8,00 \\
Fios bipolar paralelo cristal de 1{,}5mm² ou 2mm² & 1m & R\$ 7,00 \\
Conectores para canos & 1 ou mais & R\$ 5,00 à 20,00 \\
\bottomrule
\end{tabularx}

\vspace{0.5em}
{\footnotesize\textit{Fonte: Elaborado pelo autor.}\par}
\end{table}

Observa-se que na coluna “UNI” tem algo de diferente, como o dado de “1 ou mais”, que indica que a dimensão do projeto pode ser escalonada pelo produtor cliente. Abaixo é mostrado na tabela, em um grau energético, como isso seria dimensionado em capacidade de baterias e quantidade de placas solares:

\begin{table}[H]
\centering
\captionsetup{justification=centering}
\caption{\textbf{Dimensionamento energético estimado conforme a quantidade de plantas automatizadas}}
\vspace{-0.5em}
\scriptsize  % Fonte menor que footnotesize (~9pt)
\renewcommand{\arraystretch}{1.2}

\begin{tabularx}{\textwidth}{>{\centering\arraybackslash}m{1.0cm}
                                >{\centering\arraybackslash}m{1.2cm}
                                >{\centering\arraybackslash}m{1.2cm}
                                >{\centering\arraybackslash}m{2.0cm}
                                >{\centering\arraybackslash}m{1.6cm}
                                >{\centering\arraybackslash}m{1.6cm}
                                >{\centering\arraybackslash}m{1.0cm}
                                >{\centering\arraybackslash}m{1.6cm}
                                >{\centering\arraybackslash}m{1.2cm}}
\toprule
\textbf{Plantas (Kit)} & \textbf{Consumo (Ah/dia)} & \textbf{Bateria Mín. (Ah)} & \textbf{Capacidade Painel solar} & \textbf{Sobra c/ 1x 50W} & \textbf{Sobra c/ 2x 50W} & \textbf{Descarga} & \textbf{Controlador} & \textbf{Horas} \\
\midrule
1   & 5{,}21  & 10{,}42 & 1 painel de 50W   & +15{,}59 Ah & +35{,}99 Ah & 50\% & PWM 20A & 14h \\
2   & 5{,}41  & 10{,}82 & 1 painel de 50W   & +15{,}39 Ah & +35{,}79 Ah & 50\% & PWM 20A & 14h \\
10  & 7{,}21  & 14{,}42 & 1 painel de 50W   & +13{,}59 Ah & +33{,}99 Ah & 50\% & PWM 20A & 14h \\
20  & 9{,}21  & 18{,}42 & 1 painel de 50W   & +11{,}59 Ah & +31{,}99 Ah & 50\% & PWM 20A & 14h \\
30  & 11{,}21 & 22{,}42 & 2 painéis de 50W  & +9{,}59 Ah  & +29{,}99 Ah & 50\% & PWM 20A & 14h \\
40  & 13{,}21 & 26{,}42 & 2 painéis de 50W  & +7{,}59 Ah  & +27{,}99 Ah & 50\% & PWM 20A & 14h \\
50  & 15{,}21 & 30{,}42 & 2 painéis de 50W  & +5{,}59 Ah  & +25{,}99 Ah & 50\% & PWM 20A & 14h \\
75  & 20{,}21 & 40{,}42 & 2 painéis de 50W  & \textcolor{red}{-1{,}41 Ah}  & +19{,}59 Ah & 50\% & PWM 20A & 14h \\
100 & 25{,}21 & 50{,}42 & 3 painéis de 50W* & \textcolor{red}{-6{,}41 Ah}  & +14{,}59 Ah & 50\% & PWM 20A & 14h \\
120 & 29{,}21 & 58{,}42 & 3 painéis de 50W* & \textcolor{red}{-10{,}41 Ah} & +10{,}59 Ah & 50\% & PWM 20A & 14h \\
150 & 34{,}21 & 68{,}42 & 4 painéis de 50W* & \textcolor{red}{-15{,}41 Ah} & +5{,}59 Ah  & 50\% & PWM 20A & 14h \\
180 & 39{,}21 & 78{,}42 & 4 painéis de 50W* & \textcolor{red}{-20{,}41 Ah} & +0{,}59 Ah  & 50\% & PWM 20A & 14h \\
\bottomrule
\end{tabularx}

\vspace{0.5em}
{\scriptsize\textit{Fonte: Elaborado pelo autor.}\par}
\end{table}

Considerações importantes sobre a tabela de dimensionado em capacidade de baterias e quantidade de placas solares:

\begin{itemize}
  \setlength\itemsep{6pt} % Espaço entre itens
  \setlength\parskip{0pt}
  \setlength\parsep{0pt}
  \item O controlador de carga PWM de 20A foi respeitado como constante e nunca ultrapassado em termos de 240W de Corrente de carga vinda dos painéis. Capacidade segura de operação com as baterias e painéis previstos.
  
  \item Todos os consumos dos componentes foram multiplicados pelas 14 horas de funcionamento diário, exceto os casos onde o tempo de acionamento é menor (como os relés e válvulas que funcionam apenas 10 minutos por dia).

  \item A bateria mínima indicada é o dobro do consumo diário (consumo ÷ 0,5), garantindo que você não descarregue mais do que 50\% da bateria para preservar a vida útil (Regra de segurança máxima).
  
  \item Número de painéis indicado para manter folga de energia e garantir autonomia, considerando 5h de sol pleno.

  \item Sobra de energia negativa (-) indica que o painel solar não é suficiente para suprir o consumo diário sozinho.

  \item Sobra positiva (+) indica energia extra disponível para recarga e eventual segurança.

  \item A "sobra de energia" mostra quanto ainda sobraria de energia solar por dia, considerando painéis de 50W.

  \item Situações com mais de 20 kits já recomendam o uso de 2 painéis de 50W para manter folga.
  
\end{itemize}

A execução do projeto exigirá recursos humanos como mão de obra especializada. A previsão orçamentária do projeto contempla:

\listarecuada{
    \item Desenvolvimento e suporte técnico (R\$ 2.277,00).
    \item Engenheiros eletrônicos e técnicos em suporte (R\$ 2.277,00).
    \item Diretoria Executiva, Estratégica e Financeiro (R\$ 2.277,00).
    \item Capacitação e marketing (R\$ 1.518,00).
}

O investimento estimado inicial para a elaboração do projeto é de aproximadamente R\$10.000,00. Levando em consideração o acréscimo de 500 reais para a compra de alguns materiais elétricos e de estrutura. Uma vez feito, o valor poderia ser passado para o consumidor final em formato de serviço de assinatura com um valor simbólico de R\$ 99,99 onde uma parcela mensal seria posta para que seja paga mensalmente (igual streaming de vídeos).

É relevante apontar que, com o passar do tempo, o projeto fique mais acessível à medida que novos agricultores entrem no programa. O orçamento mensal considera o custo com materiais, testes, eventuais substituições de componentes e um valor reservado para apoio técnico e capacitação de usuários durante o período de validação do sistema.

\vspace*{-1.0em}  % Ajuste
\section*{\fontsize{12pt}{14pt}\selectfont\textbf{2.4. O Sistema Atual}}
\addcontentsline{toc}{section}{2.4. O Sistema Atual} 

\needspace{5\baselineskip} % reserva espaço suficiente para a próxima seção
\vspace*{-1.0em}  % Ajuste
\section*{\fontsize{12pt}{14pt}\selectfont\textbf{2.4.1. Funcionamento do sistema atual}}
\addcontentsline{toc}{section}{2.4.1. Funcionamento do sistema atual} 
\vspace{-0.7\baselineskip}

O sistema atual é uma solução automatizada de monitoramento e controle de irrigação desenvolvida com base no microcontrolador ESP32 DevKit V1, integrando sensores ambientais e atuadores com conectividade à Arduino Cloud para visualização e controle remoto.

\selectfont\textbf{Componentes e funcionamento geral:}

\listarecuada{
    \item Microcontrolador principal: ESP32, que atua como o cérebro do sistema, executando a lógica de controle, comunicação e integração com a nuvem.
    \item Placa expansora: ajuda na utilização de fios no lugar de jumpers, para contatos mais confiáveis e comunicações não duvidosas entre circuitos.
}

\selectfont\textbf{Sensores conectados:}

\listarecuada{
    \item Sensores de umidade do solo: coletam dados do nível de umidade do solo.
    \item Sensor DHT11: mede temperatura e umidade do ar.
    \item Sensor LDR: detecta luminosidade ambiente.
}

\selectfont\textbf{Atuadores e controle:}

\listarecuada{
    \item Módulos relé: acionam válvulas solenoides de 12V responsáveis pela irrigação.
    \item Válvulas solenoides: controlam o fluxo de água para irrigação, ligadas via relés.
}

\selectfont\textbf{Expansor do número de pinos analógicos/digitais:}

\listarecuada{
    \item Multiplexador CD74HC4067: permite ler múltiplos sensores de umidade usando apenas 05 portas analógicas do ESP32, otimizando recursos do microcontrolador. Pode ser ligado no máximo 6 multiplexadores, resultando em 96 sensores totais em uma ESP32. (É recomendável usar resistores pull-down).
}

\selectfont\textbf{Interface e visualização:}

\listarecuada{
    \item Display LCD I2C 16x2: exibe informações locais como valores de sensores ou status do sistema.
    \item Comunicação Bluetooth: exibe informações contínuas das leituras por intermédio de aplicativo terminal para dispositivos seriais.
    \item Arduino Cloud: permite visualizar os dados coletados e acionar manualmente a irrigação de forma remota, por meio de dashboards online.
    \item Power BI e Google Sheets: Permite a análise de dados de forma mais detalhada, uma visão mais clara e de fácil compreensão dos dados.
}

\selectfont\textbf{Conectividade:}

\listarecuada{
    \item A ESP32 conecta-se à internet via Wi-Fi usando credenciais armazenadas em \textit{arduino\_secrets.h}.
    \item As variáveis e propriedades são sincronizadas com a Arduino Cloud conforme definidas em \textit{thingProperties.h}.
    \item A assistente virtual Alexa foi integrada por meio da plataforma Arduino Cloud com ajuda de uma Skills do próprio aplicativo Alexa.
}

\selectfont\textbf{Banco de dados:}

\listarecuada{
    \item Bibliotecas de integração: o uso de credenciais para armazenamentos no google sheets com ajuda de: \textit{WiFi.h}, \textit{HTTPClient.h} e \textit{ArduinoJson.h}.
}

\selectfont\textbf{Alimentação e energia:}

\listarecuada{
    \item Fontes de 12V e 5V alimentam sensores e módulos.
    \item Há previsão de uso de painéis solares e baterias seladas para tornar o sistema autônomo energeticamente. 
    \item Economia de energia: utilização de função especial do ESP32 para diminuir o consumo de energia para o mínimo com ajuda de: \textit{esp\_sleep\_enable\_timer\_wakeup()} e \textit{esp\_deep\_sleep\_start()}.
}

\selectfont\textbf{Operação:}

O projeto utiliza uma placa microcontroladora ESP32 DevKit V1 como cérebro central do sistema. Essa placa é responsável por controlar, ler e tomar decisões com base em dados coletados de diversos sensores ambientais, integrando automação, monitoramento e eficiência energética em um único sistema de irrigação inteligente.

O sistema lê ciclicamente os sensores utilizando a função millis() (Não Bloqueante), já que o uso da função delay() (Bloqueante), não é recomendado para projetos grandes por causa do travamento das outras execuções que devem continuar funcionando; para a leitura dos dados do solo, o sistema utiliza sensores de umidade posicionados em diferentes pontos da plantação ou jardim. Como o número de sensores necessários pode ser alto, o projeto emprega multiplexadores CD74HC4067, que permitem conectar até 16 sensores analógicos em cinco entrada da ESP32. Isso amplia significativamente a quantidade de sensores que podem ser lidos, otimizando o uso dos pinos disponíveis na placa. 

Além da umidade do solo, o sistema também realiza monitoramento climático local, por meio do sensor DHT11, que mede a temperatura e a umidade do ar, e do sensor de luminosidade LDR, que identifica a intensidade da luz ambiente. Com essas informações, o sistema consegue tomar decisões mais inteligentes sobre o momento ideal para irrigar, evitando, por exemplo, ativar as válvulas durante chuvas ou períodos muito úmidos, e claro evitar irrigação em momentos de alta luminosidade, devido ao horário quente, pode acabar sendo prejudicial a planta receber água em altas temperaturas.

A irrigação é feita por meio de válvulas solenoides 12V, que são ativadas ou desativadas por módulos relé controlados pela ESP32. Cada válvula pode ser conectada a um setor ou linha de irrigação diferente. O acionamento é automatizado com base nos níveis de umidade lidos, mas pode ser ajustado ou monitorado remotamente por meio de interações no Dashboard da Arduino Cloud ou comandos de voz via Alexa. 

A alimentação elétrica do projeto é bem dimensionada para a quantidade de sensores da preferência do cliente e para proteger a bateria para ser utilizada por muito tempo, garantindo autonomia e operação off-grid, sendo distribuído por fontes de alimentação de 5V e 12V, dependendo da infraestrutura disponível. O sistema é montado em caixas de proteção IP65, para garantir durabilidade e segurança em ambientes externos.

Para otimizar ainda mais o consumo de energia, o sistema utiliza a funcionalidade de sono profundo (deep sleep) da ESP32. Essa técnica permite que o microcontrolador entre em um estado de baixo consumo elétrico quando não está realizando leituras ou acionamentos, como durante a madrugada ou entre ciclos de irrigação. Supondo que o sistema permaneça em deep sleep por 10 horas contínuas por dia, o consumo de energia da placa pode cair de cerca de 160 mA (modo ativo) para menos de 10 µA (modo deep sleep). Isso representa uma redução de mais de 99\% no consumo elétrico da ESP32 durante esse período, resultando em significativa economia de energia e maior autonomia da bateria. Em um sistema alimentado por bateria e painel solar, essa estratégia é essencial para manter o funcionamento contínuo mesmo em dias nublados ou com baixa radiação solar, reduzindo a dependência de recargas frequentes e aumentando a confiabilidade da automação em campo. 

Essa integração torna o sistema ideal para aplicações em agricultura de precisão, hortas urbanas, jardins automatizados e estudos ambientais. 

\vspace*{-1.0em}  % Ajuste
\section*{\fontsize{12pt}{14pt}\selectfont\textbf{2.4.2. Problemas do sistema atual}}
\addcontentsline{toc}{section}{2.4.2. Problemas do sistema atual} 
\vspace{-0.7\baselineskip}

Apesar de o sistema atual representar um avanço no monitoramento e controle de irrigação, ele ainda apresenta algumas limitações e desafios que justificam a busca por melhorias e por um novo sistema mais robusto e confiável:

\vspace*{0.5em}  
\selectfont\textbf{Dependência de conexão Wi-Fi local:}
\vspace*{0.5em} 

O funcionamento do sistema depende diretamente da conectividade com redes Wi-Fi configuradas previamente. Em áreas rurais ou locais com instabilidade de sinal, isso pode comprometer a atualização de dados na nuvem e o controle remoto, além do envio dos dados para o banco. 

\vspace*{0.5em} 
\selectfont\textbf{Falta de tomada de decisão inteligente:}
\vspace*{0.5em} 

O sistema atual aciona as válvulas com base em valores de sensores pré-definidos (como umidade menor que certo valor). No entanto, não há algoritmos mais complexos de inteligência artificial, aprendizado ou lógica fuzzy para adaptar a irrigação a diferentes situações climáticas, tipos de solo ou culturas específicas. 

\vspace*{0.5em} 
\selectfont\textbf{Interface limitada para o usuário local:}
\vspace*{0.5em} 

Embora haja um display LCD 16x2, ele mostra uma quantidade limitada de informações. Em caso de falha na internet, o usuário local não consegue acessar históricos, diagnósticos ou gráficos detalhados diretamente pelo sistema.

\vspace*{0.5em} 
\selectfont\textbf{Pouca modularidade e escalabilidade:}
\vspace*{0.5em} 

A estrutura atual, mesmo com o uso do multiplexador, exige ajustes manuais no código para adicionar mais sensores ou válvulas. Isso limita a escalabilidade sem intervenções técnicas constantes. 

\vspace*{0.5em} 
\selectfont\textbf{Segurança e autenticação simplificadas:}
\vspace*{0.5em} 

As credenciais Wi-Fi e o acesso à nuvem são armazenados de forma básica, sem autenticação avançada ou criptografia de dados trafegados na hora de ir para o banco por protocolo HTTP, exceto os dados que vão para a Arduino cloud por meio de protocolo MQTT, o que pode ser um problema em termos de segurança da informação. 

\vspace*{0.5em} 
\selectfont\textbf{Manutenção e diagnóstico dificultados:}
\vspace*{0.5em} 

Não há um sistema de logs ou notificações locais/online para avisar falhas (como sensor desconectado, válvula que não responde ou queda de energia). Isso dificulta a manutenção preventiva e corretiva do sistema.

\vspace*{0.5em} 
\selectfont\textbf{Ausência de armazenamento local de dados:}
\vspace*{0.5em} 

Sem conexão com a internet, o sistema não registra os dados em memória local para posterior sincronização. Isso pode gerar perda de informações importantes em caso de falhas de rede.

\vspace*{0.5em} 
\selectfont\textbf{Uso de sensores baratos de baixa precisão:}
\vspace*{0.5em} 

O sistema usa sensores de umidade do solo de baixo custo, que são acessíveis, mas apresentam leituras instáveis e pouca durabilidade. São afetados por interferências elétricas, temperatura e oxidação, exigindo calibração frequente. Isso compromete a precisão dos dados e pode levar à irrigação incorreta.

% CAPÍTULO 3 – DESENVOLVIMENTO

\needspace{5\baselineskip} % reserva espaço suficiente para a próxima seção
\vspace*{-1.0em}  % Ajuste
\section*{\fontsize{12pt}{14pt}\selectfont\textbf{3. SOLUÇÃO}}
\addcontentsline{toc}{section}{3. SOLUÇÃO}

\vspace*{-1.0em}  % Ajuste
\section*{\fontsize{12pt}{14pt}\selectfont\textbf{3.1. O sistema proposto}}
\addcontentsline{toc}{section}{3.1. O sistema proposto} 

\vspace*{-1.0em}  % Ajuste
\section*{\fontsize{12pt}{14pt}\selectfont\textbf{3.1.1. Justificativas para o novo sistema}}
\addcontentsline{toc}{section}{3.1.1. Justificativas para o novo sistema} 
\vspace{-0.7\baselineskip}

O sistema atual do agricultor apresenta diversos problemas, como a irrigação manual imprecisa, o alto consumo de água, a dependência constante de supervisão humana e o desperdício energético. Além disso, a ausência de automação dificulta o acompanhamento de variáveis importantes para o cultivo, como umidade do solo, temperatura e luminosidade.

A nova proposta resolve esses problemas ao automatizar todo o processo. A ESP32, combinada com sensores e válvulas controladas automaticamente, proporciona irrigação sob demanda, baseada em dados em tempo real. O uso de multiplexadores amplia significativamente a quantidade de sensores que o sistema pode gerenciar sem a necessidade de múltiplas placas controladoras, reduzindo custos.

A integração com painéis solares e o uso do modo de sono profundo contribuem para a sustentabilidade e eficiência energética da solução. Além disso, a comunicação com a plataforma Arduino Cloud permite o monitoramento remoto, melhorando a tomada de decisão e a segurança operacional.

\needspace{5\baselineskip} % reserva espaço suficiente para a próxima seção
\vspace*{-1.0em}  % Ajuste
\section*{\fontsize{12pt}{14pt}\selectfont\textbf{3.1.2. Situação desejada: objetivos gerais e específicos}}
\addcontentsline{toc}{section}{3.1.2. Situação desejada: objetivos gerais e específicos} 
\vspace{-0.7\baselineskip}

\selectfont\textbf{Objetivo geral:}

Desenvolver um sistema automatizado de irrigação que otimize o uso da água e da energia elétrica em pequenos e médios cultivos, com base em dados de sensores e controle remoto.

\selectfont\textbf{Objetivo específicos:}

\begin{adjustwidth}{1.2cm}{0cm}
\begin{itemize}
  \setlength\itemsep{8pt}
  \setlength\parskip{0pt}
  \setlength\parsep{0pt}

  \item Monitorar continuamente a umidade do solo em diferentes pontos usando sensores.

  \item Controlar automaticamente válvulas de irrigação conforme os níveis de umidade detectados.

  \item Ampliar a capacidade de monitoramento utilizando multiplexadores com uma única ESP32.

  \item Reduzir o consumo energético utilizando energia solar e o modo de economia da ESP32.

  \item Permitir o monitoramento remoto dos dados e do funcionamento do sistema pela internet.

  \item Utilizar sensores de baixo custo, compensando suas limitações com lógica de controle e redundância.
\end{itemize}
\end{adjustwidth}

\needspace{5\baselineskip} % reserva espaço suficiente para a próxima seção
\vspace*{-1.0em}  % Ajuste
\section*{\fontsize{12pt}{14pt}\selectfont\textbf{3.1.3. Soluções alternativas}}
\addcontentsline{toc}{section}{3.1.3. Soluções alternativas} 
\vspace{-0.7\baselineskip}

Algumas alternativas consideradas para resolver os problemas identificados foram:

\begin{enumerate}
  \renewcommand{\labelenumi}{\textbf{\Roman{enumi}.}} % Números romanos em negrito
  \setlength\itemsep{8pt}                             % Espaço entre itens

  \item \textbf{Uso de sistemas comerciais prontos de irrigação inteligente:} 
  
  Apesar de funcionais, esses sistemas costumam ter custo elevado, pouca flexibilidade e dependência de serviços proprietários.

  \item \textbf{Uso de Arduino Uno ou Mega com múltiplos sensores:}
  
  Embora viável, esses modelos não possuem conectividade Wi-Fi nativa, o que demandaria módulos adicionais e aumentaria a complexidade do sistema.

  \item \textbf{Uso de ESP32 com sensores conectados diretamente (sem multiplexador):}
  
  Essa opção limitaria drasticamente o número de sensores por conta das portas disponíveis na placa. O custo aumentaria com a adição de mais placas.
\end{enumerate}

Por fim, a solução baseada em ESP32 + multiplexadores se destacou por ser escalável, acessível, conectável à internet e eficiente no uso de recursos. Ela permite expandir o número de sensores mantendo baixo custo e complexidade reduzida, o que a torna ideal para aplicações domésticas, escolares e de agricultura urbana, que é o caso. 

\needspace{5\baselineskip} % reserva espaço suficiente para a próxima seção
\vspace*{-1.0em}  % Ajuste
\section*{\fontsize{12pt}{14pt}\selectfont\textbf{3.2. Solução escolhida}}
\addcontentsline{toc}{section}{3.2. Solução escolhida} 
\vspace{-0.7\baselineskip}

Dentre as alternativas avaliadas, a solução escolhida utiliza uma placa ESP32 DevKit V1 combinada com multiplexadores CD74HC4067, sensores de umidade de baixo custo e um sistema de controle de válvulas automatizado. A decisão levou em conta os seguintes fatores:

\selectfont\textbf{Vantagens:}

\listarecuada{
    \item Baixo custo de implementação.
    \item Conectividade Wi-Fi integrada.
    \item Capacidade de expansão via multiplexadores (16 sensores por multiplexador).
    \item Consumo reduzido de energia, com suporte a modo de sono profundo.
    \item Flexibilidade para integração com fontes solares e nós IoT.
}

\selectfont\textbf{Desvantagens:}

\listarecuada{
    \item Sensores de umidade de baixo custo têm menor precisão.
    \item Requer um pouco mais de conhecimento técnico para configuração e programação.
    \item Limitação física na expansão (número de GPIOs da ESP32).
}

A alternativa com sistemas prontos comerciais foi descartada por custo elevado e pouca flexibilidade. O uso de outras placas (como Arduino Uno/Mega) foi considerado inferior por falta de Wi-Fi nativo e maior consumo de energia.

%\needspace{5\baselineskip} % reserva espaço suficiente para a próxima seção
\vspace*{-1.0em}  % Ajuste
\section*{\fontsize{12pt}{14pt}\selectfont\textbf{3.2.1. Escopo da solução}}
\addcontentsline{toc}{section}{3.2.1. Escopo da solução} 
\vspace{-0.7\baselineskip}

\selectfont\textbf{Dentro do escopo:}

\listarecuada{
    \item Monitoramento de umidade do solo.
    \item Controle automático de válvulas de irrigação.
    \item Armazenamento e envio dos dados para a nuvem (Arduino Cloud).
    \item Modo de economia de energia (sono profundo).
    \item Alimentação por energia solar (via bateria e painel solar).
}

\selectfont\textbf{Fora do escopo:}

\listarecuada{
    \item Medidas precisas de condutividade elétrica do solo.
    \item Interface de aplicativo móvel personalizada.
    \item Suporte a fertilização automatizada.
    \item Integração com sistemas de previsão climática.
}

\needspace{5\baselineskip} % reserva espaço suficiente para a próxima seção
\vspace*{-1.0em}  % Ajuste
\section*{\fontsize{12pt}{14pt}\selectfont\textbf{3.2.2.Lista de Requisitos do Sistema}}
\addcontentsline{toc}{section}{3.2.2. Lista de Requisitos do Sistema} 
\vspace{-0.7\baselineskip}

Após análise, uma lista de requisitos funcionais e não funcionais bem completa e alinhada no sistema do código, deu origem aos seguintes requisitos:

\selectfont\textbf{Requisitos Funcionais:}

\listarecuada{
    \item O sistema deve monitorar a temperatura e a umidade relativa do ar usando um sensor DHT (ex.: DHT11/DHT22).
    \item O sistema deve monitorar a luminosidade ambiente usando um sensor LDR.
    \item O sistema deve monitorar a umidade do solo em múltiplos pontos (via multiplexador).
    \item O sistema deve acionar relés para irrigar a plantação quando a umidade do solo estiver abaixo de um valor limite.
    \item O sistema deve enviar dados de sensores periodicamente para a nuvem (Arduino IoT Cloud) via Wi-Fi.
    \item O sistema deve armazenar leituras médias de sensores e enviar à nuvem a cada 60 segundos.
    \item O sistema deve permitir depuração via BluetoothSerial, exibindo mensagens de status em tempo real.
    \item O sistema deve tentar reconectar automaticamente ao Wi-Fi em caso de perda de conexão.
    \item O sistema deve desligar a bomba de irrigação automaticamente após um tempo máximo de irrigação (para segurança).
    \item O sistema deve entrar em modo de economia de energia (deep sleep) após a irrigação do final do dia.
}

\selectfont\textbf{Requisitos Não Funcionais:}

\listarecuada{
    \item O sistema deve ser não bloqueante, utilizando \texttt{millis()} em vez de \texttt{delay()} para permitir multitarefa.
    \item O sistema deve ser compatível com a plataforma Arduino Cloud, integrando-se à infraestrutura da IoT.
    \item O sistema deve ser resiliente a falhas de rede, tentando reconectar automaticamente ao Wi-Fi.
    \item O sistema deve apresentar baixo consumo de energia, aproveitando recursos como \textit{deep sleep}.
    \item O sistema deve ser seguro, evitando sobrecarga elétrica nos relés e bombas (uso de limites máximos de tempo e proteção de hardware).
    \item O sistema deve ser fácil de configurar e permitir modificações nos parâmetros via interface web (Arduino IoT Cloud).
    \item O código deve ser modular, facilitando a manutenção futura e a expansão do projeto.
    \item O sistema deve ser compatível com a IDE Arduino e compilável na placa ESP32 DevKit v1.
    \item O sistema deve ser portátil e replicável, permitindo que outros produtores adaptem para seus pomares e plantações.
    \item O sistema deve ter uma interface de comunicação legível para depuração via BluetoothSerial.
}

%\needspace{5\baselineskip} % reserva espaço suficiente para a próxima seção
\vspace*{-1.0em}  % Ajuste
\section*{\fontsize{12pt}{14pt}\selectfont\textbf{3.2.3. Diagrama de Casos de Uso}}
\addcontentsline{toc}{section}{3.2.3. Diagrama de Casos de Uso} 
\vspace{-0.7\baselineskip}

O diagrama visa ilustrar como o Produtor Agrícola, o Arduino IoT Cloud e o Sistema de Irrigação (ESP32) interagem entre si para permitir o monitoramento das condições do pomar, o envio de dados para a nuvem e o acionamento remoto ou automático da irrigação.

O Produtor Agrícola é o usuário final, responsável por supervisionar o sistema e realizar ajustes. O Arduino IoT Cloud representa a plataforma que conecta o usuário ao sistema, possibilitando a visualização de dados e o envio de comandos. Já o Sistema de Irrigação (ESP32) representa o hardware embarcado que executa as ações automatizadas, como leitura de sensores e acionamento dos relés.

Abaixo, apresenta-se o diagrama que ilustra essas relações: 

\minhaimagemcomfonte{11cm}{capitulos/img/3_2_3_diagrama_casos_de_uso}{Diagrama de Casos de Uso do Projeto}

%\needspace{5\baselineskip} % reserva espaço suficiente para a próxima seção
\vspace*{-1.0em}  % Ajuste
\section*{\fontsize{12pt}{14pt}\selectfont\textbf{3.2.4. Especificações textuais dos casos de uso}}
\addcontentsline{toc}{section}{3.2.4. Especificações textuais dos casos de uso} 
\vspace{-0.7\baselineskip}

Para complementar o diagrama de casos de uso apresentado anteriormente, segue abaixo a descrição textual detalhando cada ator e suas respectivas funcionalidades no sistema proposto.

\selectfont\textbf{Para o Produtor Agrícola:}

\listarecuada{
    \item Visualizar Dados de Sensores: Visualizar temperatura e umidade do ar, luminosidade e umidade do solo na nuvem (via Arduino IoT Cloud).
    \item Configurar Parâmetros do Sistema: Ajustar limites de umidade, tempos de irrigação e outras configurações remotamente.
    \item Ativar/Desativar Irrigação Manualmente: Forçar o acionamento dos relés via interface da nuvem, em caso de necessidade.
}

\selectfont\textbf{Para o Arduino IoT Cloud:}

\listarecuada{
    \item Receber Dados de Sensores: Receber os valores lidos de temperatura, umidade, luminosidade e umidade do solo.
    \item Enviar Comandos ao Sistema de Irrigação: Enviar comandos para acionar ou desativar os relés, ajustar parâmetros e receber status.
}

\selectfont\textbf{Para o Sistema de Irrigação (ESP32):}

\listarecuada{
    \item Monitorar Sensores: Ler valores de temperatura e umidade do ar, luminosidade e umidade do solo.

    \item Acionar Relés de Irrigação: Acionar ou desligar relés automaticamente com base nos limites de umidade definidos.

    \item Enviar Dados para a Nuvem: Enviar leituras periódicas para o Arduino IoT Cloud.

    \item Gerenciar Conexão Wi-Fi: Monitorar a conexão Wi-Fi e tentar reconectar automaticamente em caso de falha.

    \item Entrar em Modo de Economia de Energia: Colocar o sistema em \textit{deep sleep} para poupar energia quando não estiver irrigando e na ausência de sol.
}

%\needspace{5\baselineskip} % reserva espaço suficiente para a próxima seção
\vspace*{-1.0em}  % Ajuste
\section*{\fontsize{12pt}{14pt}\selectfont\textbf{3.2.5. Modelo Conceitual de Classes}}
\addcontentsline{toc}{section}{3.2.5. Modelo Conceitual de Classes} 
\vspace{-0.7\baselineskip}

O modelo conceitual proposto reflete a organização modular do sistema de irrigação, evidenciando a separação de responsabilidades entre sensores, atuadores, lógica de controle e interface com a nuvem. Essa modelagem contribui para a manutenibilidade e a escalabilidade da solução. A seguir, é apresentada uma visão desse plano:

\minhaimagemcomfonte{16cm}{capitulos/img/3_2_5_Modelo_Conceitual_de_Classes}{Modelo Conceitual de Classes}

\needspace{5\baselineskip} % reserva espaço suficiente para a próxima seção
\vspace*{-1.0em}  % Ajuste
\section*{\fontsize{12pt}{14pt}\selectfont\textbf{3.2.6. Modelo Conceitual de Dados}}
\addcontentsline{toc}{section}{3.2.6. Modelo Conceitual de Dados} 
\vspace{-0.7\baselineskip}

\minhaimagemcomfonte{9cm}{capitulos/img/3_2_6_Modelo_Conceitual_de_Dados}{Modelo Conceitual de Dados}

\needspace{5\baselineskip} % reserva espaço suficiente para a próxima seção
\vspace*{-1.0em}  % Ajuste
\section*{\fontsize{12pt}{14pt}\selectfont\textbf{3.3. Solução Tecnológica}}
\addcontentsline{toc}{section}{3.3.	Solução Tecnológica} 

\needspace{5\baselineskip} % reserva espaço suficiente para a próxima seção
\vspace*{-1.0em}  % Ajuste
\section*{\fontsize{12pt}{14pt}\selectfont\textbf{3.3.1. Diagrama de Sequência (ou comunicação)}}
\addcontentsline{toc}{section}{3.3.1. Diagrama de Sequência (ou comunicação)} 
\vspace{-0.7\baselineskip}

\minhaimagemcomfonte{13cm}{capitulos/img/3_3_1_Diagrama_de_Sequência}{Diagrama de Sequência}

\needspace{5\baselineskip} % reserva espaço suficiente para a próxima seção
\vspace*{-1.0em}  % Ajuste
\section*{\fontsize{12pt}{14pt}\selectfont\textbf{3.3.2. Projeto de Interfaces}}
\addcontentsline{toc}{section}{3.3.2. Projeto de Interfaces} 
\vspace{-0.7\baselineskip}
% -----------------------------
\begin{figure}[H]
\begin{center}
\caption{\textbf{Dashboard de Visão Mobile}}
\includegraphics[width=13cm]{capitulos/img/Dasboard_mobile_Part_1}

%\vspace{-0.5em}
{\footnotesize\textit{Fonte: Tela do Aplicativo da Arduino Cloud (Parte 1).}\par}
\end{center}
\end{figure}
% -----------------------------
\begin{figure}[H]
\begin{center}
\caption{\textbf{Dashboard de Visão Mobile}}
\includegraphics[width=14cm]{capitulos/img/Dasboard_mobile_Part_2}

%\vspace{-0.5em}
{\footnotesize\textit{Fonte: Tela do Aplicativo da Arduino Cloud (Parte 2).}\par}
\end{center}
\end{figure}
% -----------------------------
\begin{figure}[H]
\begin{center}
\caption{\textbf{Dashboard de Visão Mobile}}
\includegraphics[width=14cm]{capitulos/img/Dasboard_mobile_Part_3}

%\vspace{-0.5em}
{\footnotesize\textit{Fonte: Tela do Aplicativo da Arduino Cloud (Parte 3).}\par}
\end{center}
\end{figure}
% -----------------------------
\begin{figure}[H]
\begin{center}
\caption{\textbf{Dashboard de Visão Mobile}}
\includegraphics[width=10cm]{capitulos/img/Dasboard_mobile_Part_4}

%\vspace{-0.5em}
{\footnotesize\textit{Fonte: Tela do Aplicativo da Arduino Cloud (Parte 4).}\par}
\end{center}
\end{figure}
% -----------------------------
\begin{figure}[H]
\begin{center}
\caption{\textbf{Dashboard de Visão Web}}
\includegraphics[width=16cm]{capitulos/img/Dasboard_mobile_Part_5}

%\vspace{-0.5em}
{\footnotesize\textit{Fonte: Tela do Dashboard da Arduino Cloud (Parte 5).}\par}
\end{center}
\end{figure}
% -----------------------------
\begin{figure}[H]
\begin{center}
\caption{\textbf{Dashboard de Visão Web}}
\includegraphics[width=16cm]{capitulos/img/Dasboard_mobile_Part_6}

%\vspace{-0.5em}
{\footnotesize\textit{Fonte: Tela do Dashboard da Arduino Cloud (Parte 6).}\par}
\end{center}
\end{figure}
% -----------------------------
\needspace{5\baselineskip} % reserva espaço suficiente para a próxima seção
\vspace*{-1.0em}  % Ajuste
\section*{\fontsize{12pt}{14pt}\selectfont\textbf{3.3.3. Diagrama de Estados}}
\addcontentsline{toc}{section}{3.3.3. Diagrama de Estados} 
\vspace{-0.7\baselineskip}

\minhaimagemcomfonte{9cm}{capitulos/img/3_3_3_Diagrama_de_Estados}{Diagrama de Estados}

\needspace{5\baselineskip} % reserva espaço suficiente para a próxima seção
\vspace*{-1.0em}  % Ajuste
\section*{\fontsize{12pt}{14pt}\selectfont\textbf{3.3.4. Diagrama de Atividades}}
\addcontentsline{toc}{section}{3.3.4. Diagrama de Atividades} 
\vspace{-0.7\baselineskip}

\minhaimagemcomfonte{7cm}{capitulos/img/3_3_4_Diagrama_de_Atividades_part_1}{Diagrama de Atividade (Parte 1)}

\minhaimagemcomfonte{5cm}{capitulos/img/3_3_4_Diagrama_de_Atividades_part_2}{Diagrama de Atividade (Parte 2)}

\minhaimagemcomfonte{6cm}{capitulos/img/3_3_4_Diagrama_de_Atividades_part_3}{Diagrama de Atividade (Parte 3)}

\minhaimagemcomfonte{5.5cm}{capitulos/img/3_3_4_Diagrama_de_Atividades_part_4}{Diagrama de Atividade (Parte 4)}

\minhaimagemcomfonte{8cm}{capitulos/img/3_3_4_Diagrama_de_Atividades_part_5}{Diagrama de Atividade (Parte 5)}

\needspace{5\baselineskip} % reserva espaço suficiente para a próxima seção
\vspace*{-1.0em}  % Ajuste
\section*{\fontsize{12pt}{14pt}\selectfont\textbf{3.3.5. Diagrama de Componentes}}
\addcontentsline{toc}{section}{3.3.5. Diagrama de Componentes} 
\vspace{-0.7\baselineskip}

\minhaimagemcomfonte{16cm}{capitulos/img/3_3_5_Diagrama_de_Componentes}{Diagrama de Componentes}

\needspace{5\baselineskip} % reserva espaço suficiente para a próxima seção
\vspace*{-1.0em}  % Ajuste
\section*{\fontsize{12pt}{14pt}\selectfont\textbf{3.3.6. Modelo de classes de Projeto}}
\addcontentsline{toc}{section}{3.3.6. Modelo de classes de Projeto} 
\vspace{-0.7\baselineskip}

\minhaimagemcomfonte{12cm}{capitulos/img/3_3_6_Modelo_de_classes_de_Projeto}{Modelo de classes de Projeto} 

\needspace{5\baselineskip} % reserva espaço suficiente para a próxima seção
\vspace*{-1.0em}  % Ajuste
\section*{\fontsize{12pt}{14pt}\selectfont\textbf{3.3.7. Modelo Físico de dados}}
\addcontentsline{toc}{section}{3.3.7. Modelo Físico de dados} 

\needspace{5\baselineskip} % reserva espaço suficiente para a próxima seção
\vspace*{-1.0em}  % Ajuste
\section*{\fontsize{12pt}{14pt}\selectfont\textbf{3.3.7.1. Projeto de Tabelas e Arquivos}}
\addcontentsline{toc}{section}{3.3.7.1. Projeto de Tabelas e Arquivos} 
\vspace{-0.7\baselineskip}

O sistema desenvolvido utiliza o Google Planilhas como meio de armazenamento dos dados sensoriais, funcionando como uma base de dados em nuvem. A planilha é alimentada automaticamente por meio de um script desenvolvido com Google Apps Script, o qual se comunica com a placa microcontroladora (ESP32) via requisições HTTP para registrar as leituras em tempo real.

A estrutura de dados adotada na planilha “planilhaDadosIOT” segue a modelagem de uma única tabela principal denominada "dados", que contém as seguintes colunas:

\begin{table}[H]
\centering
\captionsetup{justification=centering}
\caption{\textbf{Estrutura dos dados armazenados na planilha ou banco de dados IoT}}
\vspace{-0.5em}
\footnotesize
\renewcommand{\arraystretch}{1.2}

\begin{tabularx}{\textwidth}{>{\raggedright\arraybackslash}m{3.5cm}
                                >{\centering\arraybackslash}m{2.3cm}
                                >{\centering\arraybackslash}m{3cm}
                                >{\raggedright\arraybackslash}X}
\toprule
\textbf{Nome do Campo} & \textbf{Tipo de Dado} & \textbf{Formato} & \textbf{Descrição} \\
\midrule
data\_completa        & DATETIME & 2025-06-07 14:35:00 & Data e hora completa da leitura enviada pelo dispositivo. \\
data                  & DATE     & 2025-06-07           & Data extraída da leitura (sem o horário). \\
hora                  & TIME     & 14:35:00             & Horário extraído da leitura (sem a data). \\
estado\_led           & BOOLEAN  & 0 ou 1               & Estado do LED no momento da leitura. \\
estado\_rele\_A       & BOOLEAN  & 0 ou 1               & Estado do relé A. \\
estado\_rele\_B       & BOOLEAN  & 0 ou 1               & Estado do relé B. \\
temperatura           & FLOAT    & 27{,}3               & Temperatura ambiente em graus Celsius. \\
umidade\_ar           & INTEGER  & 65                   & Umidade relativa do ar (\%). \\
luminosidade          & INTEGER  & 73                   & Luminosidade ambiente (\%). \\
umidade\_solo\_3      & INTEGER  & 35                   & Umidade do solo no sensor da porta 3. \\
umidade\_solo\_10     & INTEGER  & 40                   & Umidade do solo no sensor da porta 10. \\
umidade\_solo\_14     & INTEGER  & 38                   & Umidade do solo no sensor da porta 14. \\
\bottomrule
\end{tabularx}

\vspace{0.5em}
{\footnotesize\textit{Fonte: Elaborado pelo autor.}\par}
\end{table}

\begin{itemize}
  \setlength\itemsep{8pt}
  \setlength\parskip{0pt}
  \setlength\parsep{0pt}

  \item A planilha é utilizada como um banco de dados não relacional estruturado, com linhas representando registros e colunas representando os campos do sistema

  \item A atualização dos dados é feita de forma automática e contínua, utilizando funções do Google Apps Script, que executa o tratamento dos dados recebidos e os insere na planilha.

  \item Para evitar inconsistências, o sistema realiza a separação de data e hora, além de manter um campo com a data completa (data\_completa), útil para ordenações e filtros.
\end{itemize}

A estrutura de armazenamento dos dados deste projeto foi projetada com foco em simplicidade, acessibilidade, leitura rápida e integração em tempo quase real com ferramentas de análise. Esta solução foi escolhida por sua praticidade, baixo custo (gratuita para contas Google) e facilidade de integração com outras ferramentas da Google.

Em termos de forma de armazenamento, os dados são mantidos em um formato desnormalizado, o que significa que todos os valores capturados em cada leitura são armazenados em uma única linha. Isso otimiza o desempenho de leitura, pois reduz a complexidade das consultas, tornando o acesso mais rápido — fator essencial para sistemas que priorizam a visualização e análise em tempo real.

A escalabilidade do Google Sheets também foi levada em consideração. Cada planilha pode conter até 10 milhões de células, o que, considerando uma média de 26 colunas como exemplo, permite aproximadamente 384 mil linhas de dados. Essa capacidade é suficiente para longos períodos de captação contínua sem a necessidade de segmentar ou arquivar planilhas manualmente.

Em termos de performance, os testes mostraram que o sistema leva em torno de 5 segundos para registrar 3 células, o que é considerado um tempo adequado para aplicações de sensoriamento ambiental que operam com frequência cíclica de leitura de 60 segundos.

Os dados armazenados na planilha são exportados e integrados ao Power BI \textbf{(conforme imagem em anexo I)}, que fica publicado na nuvem e gera modelos semânticos otimizados para análise. Além disso, sua hospedagem na nuvem assegura acesso remoto, segurança e compartilhamento facilitado, o que amplia a aplicabilidade dos dados captados, inclusive para decisões agronômicas, pedagógicas ou científicas.

O projeto também admite a exportação dos dados em diversos formatos, como CSV, XLSX, TSV, HTML, ODS, PDF, além do formato nativo do Google Sheets, o que facilita a portabilidade entre diferentes sistemas e plataformas.

Por fim, destaca-se que, embora o Google Sheets tenha sido a plataforma escolhida por sua simplicidade e custo zero, seria possível realizar o mesmo processo de registro utilizando outras soluções como o Google Firebase, que oferece uma base de dados NoSQL em tempo real, mais adequada para aplicações com alta frequência de leitura e escrita. Continuando a abordagem de alternativas, é possível fazer a gravação dos dados em memórias como EEPROM, cartão SD, e outros, porém essa ideia não foi explorada.

\vspace*{-1.0em}  % Ajuste
\section*{\fontsize{12pt}{14pt}\selectfont\textbf{3.3.7.2. Scripts de geração do banco e suas tabelas}}
\addcontentsline{toc}{section}{3.3.7.2. Scripts de geração do banco e suas tabelas} 
\vspace{-0.7\baselineskip}

A linguagem de programação usada nesse código é Google Apps Script, uma linguagem de script baseada em JavaScript que roda no ambiente da nuvem da Google. Ela é usada principalmente para automatizar e integrar serviços do Google, como o Google Sheets, Gmail, Google Drive, entre outros.

\captionsetup{justification=centering}
\begin{longtable}{|p{10.8cm}|p{4.1cm}|}
\caption{Trechos de Código em Google Apps Script com Suas Respectivas Ações no Sistema de Irrigação} \\
\hline
\multicolumn{1}{|c|}{\textbf{Comando em Google Apps Script}} & 
\multicolumn{1}{c|}{\textbf{Ação causada}} \\
\hline
%----------------------------------------------------------------------------
\begin{minipage}[t]{\linewidth}
\begin{lstlisting}[style=meucodigo]
function doPost(e) {
  // Abre a planilha atual
  var sheet = SpreadsheetApp.getActiveSpreadsheet();
  // Converte os dados recebidos (em texto JSON) 
  // para um objeto JavaScript
  var params = JSON.parse(e.postData.contents);
  // Identifica qual ação o ESP32 quer realizar
  var action = params.action;
  // Verifica qual ação foi pedida e chama 
  // a função correspondente
  if (action === "escreverEmLista") {
    return escreverEmLista(sheet, params);
  } else if (action === "escreverEmCelula") {
    return escreverEmCelula(sheet, params);
  } else if (action === "lerCelula") {
    return lerCelula(sheet, params);
  } else if (action === "lerLinha") {
    return lerLinha(sheet, params);
  } else {
    return ContentService.createTextOutput("Ação não reconhecida");
  }
}
\end{lstlisting}
\end{minipage}
&
Esta função é executada automaticamente quando o ESP32 envia dados via HTTP POST. \\
\hline
%----------------------------------------------------------------------------
\begin{minipage}[t]{\linewidth}
\begin{lstlisting}[style=meucodigo]
function escreverEmLista(sheet, params) {
  // Nome da aba (ex: "dados")
  var identificacao = params.identificacao;
  // Lista com os valores enviados (ex: [23.5, 70])
  var dados = params.dados;
  // Procura pela aba com o nome desejado ou  
  // cria uma nova se não existir.
  var aba = sheet.getSheetByName(identificacao)
           || sheet.insertSheet(identificacao);
  // A próxima linha vazia
  var ultimaLinha = aba.getLastRow() + 1;
  // Data e hora atuais
  var dataHora = new Date();
  aba.getRange(ultimaLinha, 1).setValue(dataHora);
  aba.getRange(ultimaLinha, 2).setValue(dataHora.toLocaleDateString());
  aba.getRange(ultimaLinha, 3).setValue(dataHora.toLocaleTimeString());

  for (var i = 0; i < dados.length; i++) {
    aba.getRange(ultimaLinha, i + 4).setValue(dados[i]);
  }

  return ContentService.createTextOutput("Dados salvos com sucesso");
}
\end{lstlisting}
\end{minipage}
&
Escreve os dados em uma nova linha na aba da planilha informada. \\
\hline
%----------------------------------------------------------------------------
\begin{minipage}[t]{\linewidth}
\begin{lstlisting}[style=meucodigo]
function escreverEmCelula(sheet, params) {
  var identificacao = params.identificacao;
  var celula = params.celula;
  var dado = params.dado;

  var aba = sheet.getSheetByName(identificacao)
           || sheet.insertSheet(identificacao);

  aba.getRange(celula).setValue(dado);
  // Escreve o dado na célula desejada
  return 
  ContentService.createTextOutput("Dado salvo na célula " + celula);
}
\end{lstlisting}
\end{minipage}
&
Escreve um único valor em uma célula específica da aba. \\
\hline
%----------------------------------------------------------------------------
\begin{minipage}[t]{\linewidth}
\begin{lstlisting}[style=meucodigo]
function lerCelula(sheet, params) {
  var identificacao = params.identificacao;
  // Nome da aba

  var celula = params.celula;
  // Exemplo: "C2"

  var aba = sheet.getSheetByName(identificacao);
  if (!aba) {
    return ContentService.createTextOutput("Aba não encontrada");
  }

  var valor = aba.getRange(celula).getValue();
  // Lê o valor da célula
  return ContentService.createTextOutput(valor);
  // Retorna o valor para o ESP32
}
\end{lstlisting}
\end{minipage}
&
Lê e retorna o valor de uma célula específica da aba. \\
\hline
%--------------------------Parte 1----------------------------------------------
\hline
\begin{minipage}[t]{\linewidth}
\begin{lstlisting}[style=meucodigo]
function lerLinha(sheet, params) {
  try {
    Logger.log("Iniciando função lerLinha");

    var identificacao = params.identificacao;
    var linha = params.linha;

    Logger.log("Identificação: " + identificacao);
    Logger.log("Linha: " + linha);

    var aba = sheet.getSheetByName(identificacao);
    if (!aba) {
      Logger.log("Aba não encontrada: " + identificacao);
      return ContentService.createTextOutput("Aba não encontrada");
    }

    Logger.log("Aba encontrada: " + identificacao);
    var ultimaLinha = aba.getLastRow();

    if (linha > ultimaLinha || linha < 1) {
\end{lstlisting}
\end{minipage}
&
Lê todos os valores de uma linha da aba, formatando a data e hora. \\
\hline
%------------------------Parte 2-----------------------------------------------
\hline
\begin{minipage}[t]{\linewidth}
\begin{lstlisting}[style=meucodigo]

      Logger.log("Linha inválida: " + linha + " (máximo permitido: " +
      ultimaLinha + ")");
      return ContentService.createTextOutput("Linha inválida: " +
      linha);
    }

    var valores 
    = aba.getRange(linha, 1, 1, aba.getLastColumn()).getValues()[0];

    var dataFormatada 
    = Utilities.formatDate(new Date(valores[0]), "GMT-3", "dd/MM/yyyy");
    var horaFormatada 
    = Utilities.formatDate(new Date(valores[1]), "GMT-3", "HH:mm:ss");

    valores[0] = dataFormatada;
    valores[1] = horaFormatada;

    Logger.log("Valores lidos da linha " + linha + ": " +
    valores.join(";"));

    return ContentService.createTextOutput(valores.join(";"));
  } catch (e) {
    Logger.log("Erro na função lerLinha: " + e.toString());
    return ContentService.createTextOutput("Erro: " + e.toString());
  }
}
\end{lstlisting}
\end{minipage}
&
(Continuação do código acima) \\
\hline

\end{longtable}
\begin{center}
\vspace{-0.5em}
{\footnotesize\textit{Fonte: Elaborado pelo autor com base no código em Google Apps Script.}\par}
\end{center}

O trecho de código abaixo está escrito em M (Power Query Formula Language), a linguagem usada pelo Power BI e pelo Excel para importar, transformar e preparar dados de várias fontes, como planilhas do Google Sheets. 

% ========================= Power Bi =======================

\captionsetup{justification=centering}
\begin{longtable}{|p{10.7cm}|p{4.2cm}|}
\caption{Trechos de Código em Linguagem M Com Suas Ações no Power BI} \\
\hline
\multicolumn{1}{|c|}{\textbf{Comando em Linguagem M}} & 
\multicolumn{1}{c|}{\textbf{Ação causada}} \\
\hline
%-------------------------------------------------
\hline
\begin{minipage}[t]{\linewidth}
\begin{lstlisting}[style=meuEstiloM]
// Define o início do código M com a palavra-chave 'let'
let
    // Passo 1: Carrega o conteúdo da planilha do Google Sheets 
    // a partir do link fornecido.
    Fonte = GoogleSheets.Contents(LinkPlanilhaEsp32),

    // Passo 2: Seleciona a aba (Sheet) chamada "dados" dentro 
    // da planilha.
    dados_Table = Fonte{[name="dados", ItemKind="Table"]}[Data],

    // Passo 3: Promove a primeira linha da tabela como cabeçalhos.
    #"Cabeçalhos Promovidos" = Table.PromoteHeaders(dados_Table,
    [PromoteAllScalars=true]),

\end{lstlisting}
\end{minipage}
&
Esse código se conecta a uma planilha do Google Sheets usando o parâmetro LinkPlanilhaEsp32. \\
\hline
%-------------------------------------------------
%-------------------------------------------------
\hline
\begin{minipage}[t]{\linewidth}
\begin{lstlisting}[style=meuEstiloM]
    // Passo 4: Altera o tipo de dados de cada coluna da tabela 
    // para garantir a consistência.
    
    // Aqui, por exemplo, definimos que a coluna "Data" é do tipo date,
    // "Hora" é do tipo time, e as demais colunas são convertidas 
    // para inteiros (Int64.Type).
    
    #"Tipo Alterado" = Table.TransformColumnTypes(
        #"Cabeçalhos Promovidos",
        {
            // Coluna com data completa
            {"Data completa", type date},  
            
            // Coluna somente com a data
            {"Data", type date},          
            
            // Coluna somente com a hora
            {"Hora", type time},          
            
            // Estado do LED como inteiro
            {"Estado do LED", Int64.Type},
            
            // Estado do relé A como inteiro
            {"Estado Rele A", Int64.Type},      
            // Estado do relé B como inteiro
            {"Estado Rele B", Int64.Type}, 
            
            // Temperatura como inteiro
            {"Temperatura (°C)", Int64.Type},   
            // Umidade como inteiro
            {"Umidade (%)", Int64.Type},
            
            // Luminosidade como inteiro
            {"Luminosidade (%)", Int64.Type},  
            
            // Umidade do solo sensor 3
            {"Umidade Solo 3", Int64.Type},     
            // Umidade do solo sensor 10
            {"Umidade Solo 10", Int64.Type},    
            // Umidade do solo sensor 14
            {"Umidade Solo 14", Int64.Type}     
        }
    )

// Define o final do código com a palavra-chave 'in' para retornar 
// o resultado final da transformação
in
    #"Tipo Alterado"
\end{lstlisting}
\end{minipage}
&
seleciona a aba "dados", promove a primeira linha como cabeçalhos e converte as colunas para tipos específicos (data, hora, inteiros). \\
\hline
%-------------------------------------------------
\end{longtable}
\begin{center}
\vspace{-0.5em}
{\footnotesize\textit{Fonte: Elaborado pelo autor com base no código.}\par}
\end{center}

\needspace{5\baselineskip} % reserva espaço suficiente para a próxima seção
\vspace*{-1.0em}  % Ajuste
\section*{\fontsize{12pt}{14pt}\selectfont\textbf{3.3.8. Ambiente tecnológico do sistema}}
\addcontentsline{toc}{section}{3.3.8. Ambiente tecnológico do sistema} 

\needspace{5\baselineskip} % reserva espaço suficiente para a próxima seção
\vspace*{-1.0em}  % Ajuste
\section*{\fontsize{12pt}{14pt}\selectfont\textbf{3.3.8.1. Ambiente Físico (diagrama de implantação)}}
\addcontentsline{toc}{section}{3.3.8.1. Ambiente Físico (diagrama de implantação)} 
\vspace{-0.7\baselineskip}

\minhaimagemcomfonte{16cm}{capitulos/img/3_3_8_1_diagrama_de_implantação}{Diagrama de Implantação}

\vspace*{-1.0em}  % Ajuste
\section*{\fontsize{12pt}{14pt}\selectfont\textbf{3.3.8.2. Justificativa da escolha da linguagem de programação)}}
\addcontentsline{toc}{section}{3.3.8.2. Justificativa da escolha da linguagem de programação} 
\vspace{-0.7\baselineskip}

O projeto em questão, que envolve sensores de solo, atuadores (como válvulas solenoides), comunicação com a nuvem e um microcontrolador ESP32, foi analisado três linguagens compatíveis e viáveis:

\vspace*{0.7em}  % Ajuste
\selectfont\textbf{1. C/C++ (Arduino)}

Apesar de antiga, mas com o baixo custo e curva de aprendizado moderada, excelente desempenho e compatibilidade total com hardware e sua alta eficiência no controle embarcado e integração com nuvem, foi a linguagem vencedora para fazer o projeto. Alguns motivos são:

\begin{longtable}{|>{\centering\arraybackslash}p{7cm}|>{\centering\arraybackslash}p{7cm}|}
\caption{Critérios Técnicos e Não Técnicos Considerados na Escolha da Linguagem C++ para o Projeto} \\
\hline
\textbf{Critérios Técnicos} & \textbf{Critérios Não Técnicos} \\
\hline
\endfirsthead
\endhead
\hline
\endfoot
\hline
\endlastfoot

Simples de escrever no contexto embarcado com uso do framework Arduino. &
Linguagem nativamente suportada pelo Arduino IoT Cloud. \\
\hline
Boa legibilidade (especialmente com bibliotecas como DHT, WiFi, ThingSpeak, etc.). &
Muito popular na comunidade maker e IoT. \\
\hline
Altamente portátil entre placas como Arduino Uno, Mega, ESP8266, ESP32. &
Alto volume de soluções prontas (exemplos, fóruns, bibliotecas). \\
\hline
Suporte completo à orientação a objetos. &
Baixo custo: IDE gratuita (Arduino IDE ou PlatformIO), com muitos tutoriais e suporte. \\
\hline
Possui recursos e bibliotecas voltadas à automação e IoT. & ~ \\
\hline
Suporta operação de baixo nível (como PWM, interrupções, etc.) fundamental para controle de sensores e atuadores. & ~ \\
\end{longtable}

\begin{center}
\vspace{-0.5em}
{\footnotesize\textit{Fonte: Elaborado pelo autor.}\par}
\end{center}

\selectfont\textbf{2. MicroPython}

Apesar de ser usada principalmente na programação de microcontroladores, dispositivos IoT (Internet das Coisas) e hardware embarcado em geral, apresentou pontos que deixou a desejar:

\begin{longtable}{|>{\centering\arraybackslash}p{7cm}|>{\centering\arraybackslash}p{7cm}|}
\caption{Critérios Técnicos e Não Técnicos Considerados na Avaliação da Linguagem Python para o Projeto} \\
\hline
\textbf{Critérios Técnicos} & \textbf{Critérios Não Técnicos} \\
\hline
\endfirsthead
\endhead
\endfoot
\endlastfoot

Simples e fácil de aprender, especialmente para iniciantes. &
Muito popular na área de ensino e prototipagem rápida. \\
\hline

Ótima legibilidade (sintaxe limpa). &
Alta produtividade (velocidade de desenvolvimento). \\
\hline

Multiplataforma. &
\textcolor{red}{Ainda pouco utilizado na Arduino IoT Cloud (não há suporte nativo direto).} \\
\hline

\textcolor{red}{Menor desempenho que C/C++ para aplicações em tempo real.} &
~ \\
\hline

\textcolor{red}{Menor disponibilidade de bibliotecas para sensores específicos comparado ao Arduino.} &
~ \\
\hline
\end{longtable}

\begin{center}
\vspace{-0.5em}
{\footnotesize\textit{Fonte: Elaborado pelo autor.}\par}
\end{center}

\selectfont\textbf{3. JavaScript}

Embora bastante versável para solucionar uma ampla gama de problemas do mundo real, também apresentou pontos que deixou a desejar:

\begin{longtable}{|>{\centering\arraybackslash}p{7cm}|>{\centering\arraybackslash}p{7cm}|}
\caption{Critérios Técnicos e Não Técnicos Considerados na Avaliação da Linguagem JavaScript para o Projeto} \\
\hline
\textbf{Critérios Técnicos} & \textbf{Critérios Não Técnicos} \\
\hline
\endfirsthead
\endhead
\endfoot
\endlastfoot

Familiar para desenvolvedores web. &
Alta popularidade na web. \\
\hline

\textcolor{red}{Pouco suporte nativo em microcontroladores de pequeno porte.} &
\textcolor{red}{Poucos desenvolvedores embarcados experientes com essa abordagem.} \\
\hline

\textcolor{red}{Baixa portabilidade para dispositivos embarcados como ESP32.} &
\textcolor{red}{Ambientes específicos necessários (como Tessel, Espruino).} \\
\hline

\textcolor{red}{Não suportada diretamente pela Arduino IoT Cloud.} &
~ \\
\hline
\end{longtable}

\begin{center}
\vspace{-0.5em}
{\footnotesize\textit{Fonte: Elaborado pelo autor.}\par}
\end{center}

Após análise dos critérios técnicos e não técnicos, a linguagem C/C++ utilizando o framework Arduino foi escolhida para este projeto com base nos seguintes motivos:
\vspace{1.5em}

\listarecuada{
    \item Compatibilidade total com o ESP32 e com a Arduino IoT Cloud, sem necessidade de adaptadores ou configurações especiais.
    \item Eficiência e controle direto do hardware, essencial para lidar com sensores, atuadores e comunicação em tempo real.
    \item Amplo suporte da comunidade, extensa documentação e bibliotecas prontas, o que reduz o tempo e o custo de desenvolvimento.
    \item Ferramentas gratuitas e maduras, como o Arduino IDE e o PlatformIO.
    \item Alta portabilidade para outros projetos embarcados com microcontroladores similares.
}

A Arduino IoT Cloud oferece suporte nativo e completo apenas para C/C++, dentro do ecossistema Arduino. Ela não oferece suporte oficial para MicroPython, JavaScript ou Python puro. Essas linguagens podem ser usadas em ambientes externos, mas com integração mais difícil ou indireta.

\needspace{5\baselineskip} % reserva espaço suficiente para a próxima seção
\vspace*{-1.0em}  % Ajuste
\section*{\fontsize{12pt}{14pt}\selectfont\textbf{3.3.8.3. Justificativa da escolha do SGBD (Sistema Gerenciador de Banco de Dados)}}
\addcontentsline{toc}{section}{3.3.8.3. Justificativa da escolha do SGBD (Sistema Gerenciador de Banco de Dados)} 
\vspace{-0.7\baselineskip}

A forma como o Google Sheets está sendo usado, é superficialmente parecido com um banco relacional por usar tabelas (linhas e colunas), mas não implementa os recursos completos de um SGBD relacional, sendo o uso de chaves primárias e estrangeiras para relacionar essas tabelas e não oferece linguagem SQL completa para manipulação relacional (embora tenha algo como QUERY()), como por exemplo, MySQL, PostgreSQL, Oracle.

O Google Sheets é mais como um repositório inicial de dados (uma “camada de coleta”), e depois importam esses dados para, se o projeto crescer muito, um banco de dados real (ex.: MySQL, PostgreSQL) para lidar com volumes maiores e garantir integridade.

Um complemento interessante abordado foi a utilização de importar dados para o Power BI, tendo muitas vantagens em desempenho e flexibilidade, especialmente para análises rápidas e relatórios interativos. Mesmo que o Power BI não seja um SGBD relacional no sentido tradicional, ele aproveita um motor de análise em memória (VertiPaq) que dá muito poder ao analista e ao usuário final.

% CAPÍTULO 4 – CONCLUSÃO

\vspace*{-1.0em} % Ajuste  
\section*{\fontsize{12pt}{14pt}\selectfont\textbf{4. CONCLUSÃO}}
\addcontentsline{toc}{section}{4. CONCLUSÃO}
\vspace{-0.7\baselineskip}

\section*{\fontsize{12pt}{14pt}\selectfont\textbf{4.1. Reflexões e comparação entre objetivos iniciais x alcançados}}
\addcontentsline{toc}{section}{4.1. Reflexões e comparação entre objetivos iniciais x alcançados}
\vspace{-0.7\baselineskip}

Ao iniciar o desenvolvimento deste Trabalho de Conclusão de Curso, estabeleceu-se como objetivo geral a criação de um sistema automatizado de irrigação capaz de otimizar o uso da água e da energia elétrica em pequenos e médios cultivos, a partir da coleta de dados de sensores e da possibilidade de controle remoto do sistema. Para viabilizar esse objetivo, foram definidos seis objetivos específicos que, em conjunto, delinearam os caminhos tecnológicos e funcionais do sistema proposto.

Após os períodos letivos de desenvolvimento, estudo técnico e modelagem de soluções, é possível realizar uma análise crítica sobre os resultados obtidos até aqui e a correspondência entre o que foi planejado e o que foi alcançado.

Embora o sistema ainda não esteja fisicamente implementado, a modelagem completa foi realizada com êxito. Foram definidas a linguagem de programação (C/C++ com Arduino), a plataforma de nuvem (Arduino IoT Cloud), o hardware central (ESP32), o uso de sensores e o ambiente operacional. A solução proposta é viável tecnicamente e condiz com o objetivo inicial de otimizar recursos naturais e possibilitar controle remoto, ou seja, o objetivo geral foi atendido em sua totalidade dentro do escopo de projeto e modelagem. 

Durante o desenvolvimento do projeto, os objetivos específicos foram majoritariamente alcançados conforme o planejado. O sistema foi modelado para monitorar continuamente a umidade do solo em diferentes pontos, utilizando sensores de baixo custo conectados à ESP32, com possibilidade de expansão. A lógica de controle automático das válvulas de irrigação com base nos níveis de umidade foi definida e a ESP32 está preparada para realizar essa função de forma autônoma. Também foi implementada a ampliação da capacidade de monitoramento por meio do uso de multiplexadores, o que permite a leitura de múltiplos sensores com apenas uma placa controladora. Quanto ao consumo energético, embora o uso de energia solar tenha sido considerado e o modo de economia da ESP32 estudado e implementado, essa parte ainda depende de testes físicos com painéis solares, que acabou não sendo plenamente implementada até o momento devido aos recursos fiscais baixos, entretanto foi feito cálculos com base em dados matemáticos, levando em consideração consumo de equipamentos e produção de energia que pode ser visto no item 2.3.4. O sistema também foi modelado para permitir o monitoramento remoto via internet, com apoio da plataforma Arduino IoT Cloud, garantindo acesso aos dados em tempo real e controle remoto. Por fim, a escolha por sensores acessíveis foi compensada com o uso de lógica de controle e redundância, assegurando maior confiabilidade nas leituras mesmo com dispositivos de menor precisão.

\vspace*{-1.0em} % Ajuste  
\section*{\fontsize{12pt}{14pt}\selectfont\textbf{4.2. Vantagens e desvantagens do sistema}}
\addcontentsline{toc}{section}{4.2. Vantagens e desvantagens do sistema}
\vspace{-0.7\baselineskip}

O sistema automatizado de irrigação proposto apresenta uma série de vantagens que o tornam uma solução viável, eficiente e adaptável para pequenos e médios produtores. Dentre os principais benefícios, destaca-se a eficiência no uso da água, uma vez que o sistema realiza a irrigação de maneira inteligente, apenas quando os sensores detectam níveis críticos de umidade no solo e em momentos que o tempo não está muito quente para não prejudicar a planta. Isso evita o desperdício de água e danos aos tecidos da árvore, respectivamente, e claro promove o uso racional dos recursos hídricos e qualidade a planta. Além disso, o sistema foi pensado para operar com baixo consumo energético, utilizando a funcionalidade de economia de energia da ESP32 e prevendo a integração com energia solar, o que garante uma operação sustentável e de baixo custo a longo prazo.

A escolha de componentes de baixo custo, como sensores analógicos de umidade e multiplexadores, permite que a solução seja financeiramente acessível, mesmo para produtores com recursos limitados. Outro ponto forte é a capacidade de monitoramento remoto por meio da plataforma Arduino IoT Cloud, que possibilita o acesso aos dados e o controle do sistema pela internet, oferecendo comodidade e controle ao usuário, mesmo à distância. O projeto também é escalável, permitindo a expansão do número de sensores sem a necessidade de novas placas, graças ao uso de multiplexadores CD74HC4067. A facilidade de manutenção e atualização do sistema, garantida pela ampla documentação da linguagem C/C++ e da comunidade Arduino, também contribui para sua robustez e longevidade.

Por outro lado, o sistema também apresenta algumas desvantagens que merecem ser consideradas. A dependência de conexão Wi-Fi pode ser um fator limitante em áreas rurais onde a cobertura de internet é instável ou inexistente. Nesse caso, seriam necessárias soluções complementares, como comunicação por rádio (LoRa) ou armazenamento local de dados. Outro ponto delicado está na precisão dos sensores analógicos de umidade, que, por serem de baixo custo, podem apresentar variações nas leituras e menor durabilidade, exigindo estratégias como lógica de redundância ou calibração periódica.

A dependência da energia solar também impõe um desafio adicional, já que a eficiência do sistema pode ser comprometida em dias nublados ou em regiões com baixa insolação, o que requer um bom planejamento do sistema de energia. Além disso, a montagem e instalação do sistema exigem conhecimentos básicos de eletrônica e redes, o que pode representar uma barreira para usuários sem formação técnica. Embora o custo dos componentes eletrônicos seja baixo, o investimento inicial em painéis solares e baterias pode ser significativo. Por fim, a capacidade de processamento e memória da ESP32 pode limitar a inclusão de funcionalidades mais avançadas, especialmente em cenários com um número elevado de sensores ou necessidade de processamento local complexo, isso pode fazer o uso de mais de uma ESP32.

Apesar dessas limitações, os benefícios superam as desvantagens, especialmente quando se considera a proposta do projeto de oferecer uma solução acessível, eficiente e sustentável para automação da irrigação. A escolha dos componentes e a modelagem do sistema foram feitas com foco na viabilidade prática, buscando o melhor equilíbrio entre custo e benefício.

\vspace*{-1.0em} % Ajuste  
\section*{\fontsize{12pt}{14pt}\selectfont\textbf{4.3. Trabalhos futuros}}
\addcontentsline{toc}{section}{4.3. Trabalhos futuros}
\vspace{-0.7\baselineskip}

Uma das principais possibilidades está na implementação efetiva do sistema de alimentação por energia solar, com o dimensionamento ideal de painéis e baterias, testes práticos de autonomia e mecanismos de proteção e monitoramento de carregamento inteligente, o que tornará o sistema mais autossuficiente e sustentável. 

Outra frente importante para evolução é o aprimoramento do modelo de controle de irrigação, incorporando variáveis climáticas externas, como temperatura e umidade do ar, previsão do tempo, por meio de APIs meteorológicas. Isso permitiria um sistema ainda mais inteligente, capaz de antecipar ou adiar irrigações com base em condições ambientais previstas, otimizando ainda mais o uso da água. 

Também se identifica como relevante a possibilidade de desenvolver uma solução para situações em que o acesso à internet é limitado ou inexistente, futuros desenvolvimentos poderiam explorar a comunicação via rádio frequência (como LoRa) ou a armazenagem local de dados em cartões SD, garantindo o funcionamento contínuo do sistema mesmo em regiões remotas. 

A integração do sistema de irrigação com outras soluções agrícolas, como monitoramento de pragas e qualidade das plantas por processamento de imagem, controle de temperatura (Caso seja irrigação em estufas) ou automação de fertilização (fertirrigação), pode formar um ecossistema agrícola inteligente, potencializando ainda mais a produtividade e sustentabilidade das plantações. Esses avanços, embora não incorporados, representam um caminho natural de evolução e podem ser explorados em projetos futuros, contribuindo para um modelo de agricultura de precisão acessível, eficiente e escalável. 

Além das possibilidades já previstas, outras ideias promissoras podem ser exploradas em versões futuras do sistema. A adição de sensores como o de chuva, fluxo de água e proximidade da caixa d’água pode refinar ainda mais o controle e a automação. Recursos como o sensor RTC (Relógio de Tempo Real trariam maior precisão e histórico ao sistema. A instalação de LEDs de indicação pode ajudar na sinalização visual de falhas ou estados de funcionamento.

Por fim, para ampliar a conectividade, o uso de um módulo com chip de internet permitiria comunicação em áreas sem Wi-Fi. Também se vislumbra a criação de um aplicativo exclusivo chamado Deméter, em homenagem à deusa da agricultura, e a aplicação de inteligência artificial tanto para interações com o agricultor quanto para prever necessidades de irrigação. Outras ideias incluem o uso de sensores a laser com LDR para proteção da plantação contra animais e a implementação de bombas solares, reforçando o compromisso com soluções sustentáveis e inteligentes para o campo. 

\vspace*{-1.0em} % Ajuste  

\section*{\fontsize{12}{14}\selectfont\textbf{5. REFERÊNCIAS BIBLIOGRÁFICAS}}
\vspace*{-0.7\baselineskip}  % Remove o espaço extra entre título e texto
\addcontentsline{toc}{section}{5. REFERÊNCIAS BIBLIOGRÁFICAS}

{\setlength{\parindent}{0pt}
\justifying

\noindent
\textbf{Repositórios no GitHub e arquivos brutos}
\vspace{0.8em}

\noindent
ARCHIVE.ORG. \textit{Fritzing 0.9.3b Windows (64-bit) zip}. Disponível em: \url{https://archive.org/download/fritzing.0.9.3b.64.pc/fritzing.0.9.3b.64.pc.zip}. Acesso em: 08 jun 2025.
\vspace{1.5em}

\noindent
GITHUB (arduino). \textit{Arduino Create Agent – releases}. Disponível em: \url{https://github.com/arduino/arduino-create-agent/releases}. Acesso em: 08 jun. 2025.
\vspace{1.5em}

\noindent
GITHUB (bblanchon). \textit{ArduinoJson}. Disponível em: \url{https://github.com/bblanchon/ArduinoJson#}. Acesso em: 08 jun. 2025.
\vspace{1.5em}

\noindent
GITHUB (beegee‑tokyo). \textit{DHTesp: ESPx library for DHT sensors}. Disponível em: \url{https://github.com/beegee-tokyo/DHTesp}. Acesso em: 08 jun. 2025.
\vspace{1.5em}

\noindent
GITHUB (thalestgf). \textit{Código do Apps Script – Google‑planilhas}. Disponível em: \url{https://github.com/thalestgf/Google-planilhas/blob/main/C%C3%B3digo%20do%20Apps%20Script}. Acesso em: 08 jun. 2025.
\vspace{1.5em}

\noindent
RAW.GITHUBUSERCONTENT. \textit{package\_esp32\_index.json (Espressif / Arduino ESP32)}. Disponível em: \url{https://raw.githubusercontent.com/espressif/arduino-esp32/gh-pages/package_esp32_index.json}. Acesso em: 08 jun. 2025.
\vspace{1.5em}

\noindent
\textbf{Drivers e ferramentas}
\vspace{0.8em}

\noindent
DIGILOG.PK. \textit{CH9102X serial‑chip: features, drivers, applications}. Disponível em: \url{https://digilog.pk/blogs/news/ch9102x-serial-chip-features-drivers-applications?srsltid=AfmBOopggmZRPVZmdrax335DyW07mHYgCRnSTMoVr31VyA5YhoxqLMeW}. Acesso em: 08 jun. 2025.
\vspace{1.5em}

\noindent
RANDOM NERD TUTORIALS. \textit{Install ESP32/ESP8266 USB drivers (CP210x) – Windows}. Disponível em: \url{https://randomnerdtutorials.com/install-esp32-esp8266-usb-drivers-cp210x-windows/}. Acesso: 08 jun. 2025.
\vspace{1.5em}

\noindent
SILABS. \textit{USB to UART Bridge (VCP) – drivers}. Disponível em: \url{https://www.silabs.com/developer-tools/usb-to-uart-bridge-vcp-drivers}. Acesso em: 08 jun. 2025.
\vspace{1.5em}

\noindent
\textbf{Documentações e artigos}
\vspace{0.8em}

\noindent
SUPPORT.ARDUINO.CC. \textit{Artigo HC360014869820}. Disponível em: \url{https://support.arduino.cc/hc/en-us/articles/360014869820}. Acesso em: 08 jun. 2025.
\vspace{1.5em}

\noindent
DOCS.ARDUINO.CC. \textit{Arduino Cloud – guia para ESP32}. Disponível em: \url{https://docs.arduino.cc/arduino-cloud/guides/esp32/}. Acesso em: 08 jun. 2025.
\vspace{1.5em}

\noindent
BLOG.ELETROGATE. \textit{Conhecendo o ESP32 – introdução}. Disponível em: \url{https://blog.eletrogate.com/conhecendo-o-esp32-introducao-1/}. Acesso em: 08 jun. 2025.
\vspace{1.5em}

\noindent
RANDOM NERD TUTORIALS. \textit{ESP32 deep‑sleep + wake‑up sources}. Disponível em: \url{https://randomnerdtutorials.com/esp32-deep-sleep-arduino-ide-wake-up-sources/}. Acesso em: 08 jun. 2025.
\vspace{0.5em}

\noindent
CRESCER ENGENHARIA. \textit{Como usar o ESP32 para publicar dados no Google Sheets}. Disponível em: \url{https://www.crescerengenharia.com/post/aprenda-como-usar-o-esp32-para-publicar-dados-no-google-sheets}. Acesso em: 08 jun. 2025.
\vspace{1.5em}

\noindent
ESP32IO. \textit{Tutorial – ESP32 Relay}. Disponível em: \url{https://esp32io.com/tutorials/esp32-relay}. Acesso em: 08 jun. 2025.
\vspace{1.5em}

\noindent
PLANTUML. \textit{PlantUML – Visualização UML}. Disponível em: \url{https://plantuml.com/}. Acesso em: 08 jun. 2025.
\vspace{1.5em}

\noindent
MAKERHERO. \textit{Substituindo delay por millis no Arduino}. Disponível em: \url{https://www.makerhero.com/blog/subtituindo-delay-por-millis-no-arduino/}. Acesso em: 08 jun. 2025.
\vspace{1.5em}

\noindent
MAKERHERO. \textit{Arduino Cloud – guia completo para projetos IoT}. Disponível em: \url{https://www.makerhero.com/blog/arduino-cloud-guia-completo-para-projetos-iot/}. Acesso em: 08 jun. 2025.
\vspace{1.5em}

\noindent
DOCS.ARDUINO.CC. \textit{Função millis – referência da linguagem}. Disponível em: \url{https://docs.arduino.cc/language-reference/fun%C3%A7%C3%B5es/time/millis/}. Acesso em: 08 jun. 2025.
\vspace{1.5em}

\noindent
DOCS.ARDUINO.CC. \textit{Função delay – referência da linguagem}. Disponível em: \url{https://docs.arduino.cc/language-reference/fun%C3%A7%C3%B5es/time/delay/}. Acesso em: 08 jun. 2025.
\vspace{1.5em}

\noindent
DOCS.ARDUINO.CC. \textit{Função random – referência da linguagem}. Disponível em: \url{https://docs.arduino.cc/language-reference/pt/fun%C3%A7%C3%B5es/random-numbers/random/}. Acesso em: 08 jun. 2025.
\vspace{1.5em}

\noindent
\textbf{Fóruns e hub}
\vspace{0.8em}

\noindent
FÓRUM ARDUINO. \textit{Discussão: falar com o pessoal da Arduino Cloud}. Disponível em: \url{https://forum.arduino.cc/t/falar-com-o-pessoal-da-arduino-cloud/1184940}. Acesso em: 08 jun. 2025.
\vspace{1.5em}

\noindent
FÓRUM ARDUINO. \textit{Discussão – serial-port COM12 fatal error}. Disponível em: \url{https://forum.arduino.cc/t/serial-port-com12-a-fatal-error-occurred-could-not-open}
\url{-com12-the-port-doesnt-exist-executing-command-exit-status-2/1107334/7}. Acesso em: 08 jun. 2025.
\vspace{1.5em}

\noindent
PROJECT HUB (Arduino). \textit{Trending projects}. Disponível em: \url{https://projecthub.arduino.cc/trending}. Acesso em: 08 jun. 2025.
\vspace{1.5em}

\noindent
\textbf{Software oficial}
\vspace{0.8em}

\noindent
ARDUINO.CC. \textit{Página – Software}. Disponível em: \url{https://www.arduino.cc/en/software/}. Acesso em: 08 jun. 2025.
\vspace{1.5em}

\noindent
\textbf{Vídeos no YouTube}
\vspace{0.8em}

\noindent
YOUTUBE. Disponível em: \url{https://www.youtube.com/watch?v=b9Qn7UmaTiA}. Acesso em: 08 jun. 2025.
\vspace{1.5em}

\noindent
YOUTUBE. Disponível em: \url{https://www.youtube.com/watch?v=DfU6llvIMcM}. Acesso em: 08 jun. 2025.
\vspace{1.5em}

\noindent
YOUTUBE. Disponível em: \url{https://www.youtube.com/watch?v=xyIFOVW-ar0}. Acesso em: 08 jun. 2025.
\vspace{1.5em}

\noindent
YOUTUBE. Disponível em: \url{https://www.youtube.com/watch?v=xmiL49UjfzI}. Acesso em: 08 jun. 2025.
\vspace{1.5em}

\noindent
YOUTUBE. Disponível em: \url{https://www.youtube.com/watch?v=RZxJDC8YhO0}. Acesso em: 08 jun. 2025.
\vspace{1.5em}

\noindent
YOUTUBE. Disponível em: \url{https://www.youtube.com/watch?v=666uZTCcRGk}. Acesso em: 08 jun. 2025.
\vspace{1.5em}

\noindent
YOUTUBE. Disponível em: \url{https://www.youtube.com/watch?v=gNv8tzyb0BU}. Acesso em: 08 jun. 2025.
\vspace{1.5em}

\noindent
YOUTUBE. Disponível em: \url{https://www.youtube.com/watch?v=n7EqnUSmR70}. Acesso em: 08 jun. 2025.
\vspace{1.5em}

\noindent
YOUTUBE. Disponível em: \url{https://www.youtube.com/watch?v=i7aHGXylb-w}. Acesso em: 08 jun. 2025.
\vspace{1.5em}

\noindent
YOUTUBE. Disponível em: \url{https://www.youtube.com/watch?v=5OSPk5oHhVM}. Acesso em: 08 jun. 2025.
\vspace{1.5em}

\noindent
YOUTUBE. Disponível em: \url{https://www.youtube.com/watch?v=fwkhcbPyLQE}. Acesso em: 08 jun. 2025.
\vspace{1.5em}

\noindent
YOUTUBE. Disponível em: \url{https://www.youtube.com/watch?v=6iDfsM9YhCQ}. Acesso em: 08 jun. 2025.
\vspace{1.5em}

\noindent
YOUTUBE. Disponível em: \url{https://www.youtube.com/watch?v=khRBvu15Rvs}. Acesso em: 08 jun. 2025.
\vspace{1.5em}

\noindent
YOUTUBE. Disponível em: \url{https://www.youtube.com/watch?v=Q6u_1sPOeTs}. Acesso em: 08 jun. 2025.
\vspace{1.5em}

\noindent
YOUTUBE. Disponível em: \url{https://www.youtube.com/watch?v=yXjlJ1CRHpQ}. Acesso em: 08 jun. 2025.
\vspace{1.5em}

\noindent
YOUTUBE. Disponível em: \url{https://www.youtube.com/watch?v=44S61HZOJ78}. Acesso em: 08 jun. 2025.
\vspace{1.5em}

\noindent
YOUTUBE. Disponível em: \url{https://www.youtube.com/watch?v=gq1-G3Lbym8}. Acesso em: 08 jun. 2025.
\vspace{1.5em}

\noindent
YOUTUBE. Disponível em: \url{https://www.youtube.com/watch?v=44mnBnOJXUI}. Acesso em: 08 jun. 2025.
\vspace{1.5em}

\noindent
YOUTUBE. Disponível em: \url{https://www.youtube.com/watch?v=O82oahVI08w}. Acesso em: 08 jun. 2025.
\vspace{1.5em}

\noindent
YOUTUBE. Disponível em: \url{https://www.youtube.com/watch?v=6V2usR394nc}. Acesso em: 08 jun. 2025.
\vspace{1.5em}

\noindent
YOUTUBE. Disponível em: \url{https://www.youtube.com/watch?v=0YdvKucO9dU}. Acesso em: 08 jun. 2025.
\vspace{1.5em}

\noindent
YOUTUBE. Disponível em: \url{https://www.youtube.com/watch?v=JtttnL28m3Q}. Acesso em: 08 jun. 2025.
\vspace{1.5em}

\noindent
YOUTUBE. Disponível em: \url{https://www.youtube.com/watch?v=1WIWrmc-rBk}. Acesso em: 08 jun. 2025.
\vspace{1.5em}

\noindent
YOUTUBE. Disponível em: \url{https://www.youtube.com/watch?v=N0MAJY-gI3E}. Acesso em: 08 jun. 2025.
\vspace{1.5em}

\noindent
YOUTUBE. Disponível em: \url{https://www.youtube.com/watch?v=C54Cp819Ebc}. Acesso em: 08 jun. 2025.
\vspace{1.5em}

\noindent
YOUTUBE. Disponível em: \url{https://www.youtube.com/watch?v=MGbgyuVXRcI}. Acesso em: 08 jun. 2025.
\vspace{1.5em}

\noindent
YOUTUBE. Disponível em: \url{https://www.youtube.com/watch?v=tvR1yLMx6fo}. Acesso em: 08 jun. 2025.
\vspace{1.5em}

\noindent
YOUTUBE. Disponível em: \url{https://www.youtube.com/watch?v=OMvZjwFYimo}. Acesso em: 08 jun. 2025.
\vspace{1.5em}

\noindent
YOUTUBE. Disponível em: \url{https://www.youtube.com/watch?v=YSVoxUxyASw}. Acesso em: 08 jun. 2025.
\vspace{1.5em}

\noindent
\textbf{Relatórios e visualizações}
\vspace{0.8em}

\sloppy
\noindent
LOPES, André Oliveira. \textit{RELATÓRIO – DADOS DO TCC}. Disponível em: \url{https://app.powerbi.com/view?r=eyJrIjoiM2VlYjZkMzAtNTUwOC00YzVlLTk5YWYtZGI1YWM3N}
\url{mQyZmRjIiwidCI6IjY5ZDliOGMyLWU0MjUtNDNjZS04N2UxLTYzZTQyZTViODViY} 
\url{yJ9}. Acesso em: 08 jun. 2025.
\vspace{1.5em}

}

% Anexos 

\needspace{5\baselineskip} % reserva espaço suficiente para a próxima seção
\vspace*{-3.0em} % Ajuste  
{\centering
  \section*{\fontsize{12pt}{14pt}\selectfont\textbf{Anexo I}}
\par}
\addcontentsline{toc}{section}{Anexo I}
\vspace{-0.7\baselineskip}

\minhaimagemcomfonte{12cm}{capitulos/img/app-powerbi-view}{Dashboard de visualização de dados} 

\vspace*{-1.0em} % Ajuste  
{\centering
  \section*{\fontsize{12pt}{14pt}\selectfont\textbf{Anexo II}}
\par}
\addcontentsline{toc}{section}{Anexo II}
\vspace{-0.7\baselineskip}

\minhaimagemcomfonte{12cm}{capitulos/img/Prototipagem_do_sistema}{Prototipagem do sistema feito em Fritzing} 


\end{document}
